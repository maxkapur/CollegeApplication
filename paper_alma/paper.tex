%!TEX encoding = UTF-8 Unicode 

\documentclass[11pt]{article} %try amsproc, amsart
\usepackage{geometry}       \geometry{a4paper, margin=2.5cm}
\usepackage{setspace}       \setstretch{1.1}
\usepackage{hyperref}       \hyperbaseurl{}    
                            \urlstyle{same}
\usepackage[ruled,linesnumbered]{algorithm2e}   \SetKwInOut{KwParams}{Parameters}
                                                \newcommand{\lIfElse}[3]{\lIf{#1}{#2 \textbf{else}~#3}}
\usepackage{stackengine}    \newcommand\xrowht[2][0]{\addstackgap[.5\dimexpr#2\relax]{\vphantom{#1}}}
\usepackage[style=numeric]{biblatex}
\addbibresource{refs.bib}

% Author block
\usepackage{authblk}

\title{A quadratic-time algorithm for the cardinality-constrained college application problem}
\author[1]{Max Kapur}
\author[2]{Sung-Pil Hong}
\affil[1,2]{Department of Industrial Engineering, Seoul National University}
\date{\today}

% Math
\usepackage{amssymb}
\usepackage{amsmath}
\usepackage{amsthm}

\DeclareMathOperator*{\argmax}{arg\,max}
\DeclareMathOperator*{\argmin}{arg\,min}

\newtheorem{theorem}{Theorem}
\newtheorem{lemma}{Lemma}
\newtheorem{corollary}{Corollary}
\newtheorem{proposition}{Proposition}
\theoremstyle{definition}
\newtheorem{example}{Example}
\newtheorem{definition}{Definition}
\newtheorem{problem}{Problem}
\newtheorem{assumption}{Assumption}
\newtheorem{conjecture}{Conjecture}

%\numberwithin{equation}{section}
%\numberwithin{theorem}{section}
%\numberwithin{lemma}{section}
%\numberwithin{corollary}{section}
%\numberwithin{proposition}{section}
%\numberwithin{example}{section}
%\numberwithin{definition}{section}
%\numberwithin{problem}{section}
%\numberwithin{assumption}{section}
%\numberwithin{conjecture}{section}


\begin{document}

\maketitle


\begin{abstract}
%This paper considers a novel portfolio optimization problem called the college application
problem. We show that the objective function is a nondecreasing submodular set function
and provide a quadratic-time algorithm for its maximization over a cardinality
constraint. The algorithm's validity rests on the nested structure of the optimal solution,
a property that is of independent interest. Further analysis of the optimal application strategy
helps explain perplexing findings from regression studies of college applicant behavior:
In our model, systematic differences in the risk allocations of low- and high-income students
emerge spontaneously from differences in applicants' budgets, even in the absence of risk aversion.
Also, our analysis suggests that the underwhelming effectiveness of behavioral nudges
(such as application fee waivers) designed to encourage talented low-income students to apply
to more colleges can be attributed to the fact that these interventions have typically targeted
students with small shadow costs.

This paper considers a novel portfolio optimization problem called the college application problem. We show that the objective function is a nondecreasing submodular set function and provide quadratic-time solution algorithm. We discuss the properties of the optimal solution and their implications for student welfare.

\textbf{Keywords:} combinatorial optimization, submodular maximization, portfolio optimization

\vfill
Correspondence may be addressed to Max Kapur.
\begin{itemize}
%\item[] Correspondence may be addressed to Max Kapur.

\item[] Email: \url{maxkapur@gmail.com}

\item[] Address: 39-411, 1 Gwanak-ro, Gwanak-gu, Seoul 08826, Republic of Korea
\end{itemize}

\end{abstract}


\pagebreak

\tableofcontents

%\section{Introduction}

Due to the widespread perception that career earnings depend on college admissions outcomes, the solution of \eqref{headlineproblem} holds monetary value. In the American college consulting industry, students pay private consultants an average of \$200 per hour for assistance in preparing application materials, estimating their admissions odds, and identifying target schools \cite{sklarow2018}. In Korea, an important revenue stream for admissions consulting firms such as Megastudy (\url{megastudy.net}) and Jinhak (\url{jinhak.com}) is ``mock application'' software that claims to use artificial intelligence to optimize the client's application strategy. 

Most quantitative research on college admissions to date has been descriptive in nature. A standard benchmark for logistic regression techniques, for example, involves estimating students' admissions probabilities from information about their past academic performance \cite{acharyaetal2019,lim2013}. When professional admissions consultants advise their clients on where to apply to college, however, they typically rely on qualitative heuristics, such as balancing applications between safety and reach schools, rather than optimization models \cite{jeon2015,peck2021}. (Admissions consultants refer to schools with low utility and high admissions probability as \emph{safety schools}, and those with high utility and low admissions probability as \emph{reach schools.})

In the present study, we formulate the student's college application decision as the following combinatorial optimization problem:
\begin{align} \label{headlineproblem}
\begin{split}
\text{maximize}\quad & v(\mathcal{X}) =  \operatorname{E}\Bigl[\max\bigr\{t_0,
\max\{t_j Z_j : j \in \mathcal{X}\}\bigr\}\Bigr] \\
\text{subject to}\quad & \mathcal{X} \subseteq \mathcal{C}, ~~|\mathcal{X}| \leq H
\end{split}
\end{align}
Here $\mathcal{C} = \{ 1 \dots m\}$ represents the set of colleges in the market and $h > 0$ is the number of colleges the student can apply to. For $j = 1 \dots m$, let $f_j$ denote the student's probability of being admitted to school $j$ if she applies and $Z_j$ a random, independent Bernoulli variable with probability $f_j$, so that $Z_j$ equals one if she is admitted and zero if not. It is appropriate to assume that the $Z_j$ are statistically independent as long as $f_j$ are probabilities estimated specifically for this student (as opposed to generic acceptance rates). For $j = 0\dots m$, $t_j\geq 0$ indicates the utility she associates with attending school $j$, where her utility is $t_0$ if she does not attend college. Then the student's objective is to maximize the expected utility associated with the best school she is admitted to. Therefore, her optimal college application strategy is given by the solution $\mathcal{X}$ to the problem above, where $\mathcal{X}$ represents the set of schools to which she applies, which we call her application \emph{portfolio.}

The problem can also be applied to other competitive matching games such as job application or marriage proposal. Here the budget constraint may arise due to the time required to communicate with each applicant or a legal limit on the number of applications allowed.

We show that the optimal portfolios for the cardinality-constrained college are \emph{nested} in the budget constraint, meaning that when $\mathcal{X}_h$ is the optimal portfolio of size $h$, $\mathcal{X}_h \subset \mathcal{X}_{h+1}$. The nestedness property is equivalent to the optimality of the \emph{greedy} algorithm that iteratively adds to $\mathcal{X}$ the school that elicits the greatest increase in the objective value. The proof of the nestedness property yields insight into the properties of the optimal college application strategy: Students who cannot afford to apply to many schools allocate a greater proportion of their application budget to safety schools than students with a large application budget.% rewrite: A classic finding of behavioral psychology is that agents with limited resources tend to make more risk-averse decisions than their wealthier peers, especially with respect to education and career choice \cite{hartlaubandschneider2012,vanhuizenandalessie2019}. In the case of college application, our model (though theoretical) provides a rational basis for this tendency.

Our analysis also provides tight bounds on the marginal value of additional applications, which may be informative in the design of fee waiver policies and setting application markets in centralized admissions markets.
% discuss below
%Our analysis also reveals that the optimal objective value grows sublinearly in $h$, a fact which may be useful in designing admissions markets. For example, in the South Korean college admissions process, students are allowed to apply to only $h = 3$ colleges in the main application cycle, a policy which is thought to promote fairness. Therefore, by submitting an additional application, no student can increase her expected utility by more than 33 percent, 

% \subsection{Literature review}

\subsection{Related optimization problems}

The objective function is a nondecreasing, submodular set function in the sense first described by Nemhauser et al. \cite{nemhauseretal1978}. They showed that the greedy algorithm is asymptotically $(1 - 1/e)$-optimal for optimizing a monotonic submodular function over a cardinality constraint, and a subsequent result of Nemhauser and Wolsey \cite{nemhauserandwolsey1978} proved that this approximation ratio is the best achievable by a polynomial-time algorithm. In the special case of college application, however, we show that the same algorithm is exact. 

% define nestedness
The proof of the optimality of the greedy algorithm rests on the nested structure of the portfolios, a property that is of independent interest. As Rozanov and Tamir \cite{rozanovandtamir2020} articulate, the knowledge that the optima are nested aids not only in computing the optimal solution, but in the implementation thereof under uncertain information. For example, in the United States, many college applications are due at the beginning of November, and it is typical for students to begin working on their applications during the prior summer because colleges reward students who tailor their essays to the target school. However, students may not know how many schools they can afford to apply to until late October. The nestedness property---or equivalently, the validity of a greedy algorithm---implies that even in the absence of complete budget information, students can begin to carry out the optimal application strategy by writing essays for schools in the order that they enter the optimal portfolio.
% Greedy algorithms produce a \emph{nested} family of solutions parameterized by the budget $H$: If $H \leq H'$, then the greedy solution for budget $H$ is a subset of the greedy solution for budget $H'$.

% Reword next two paragraphs to prevent collision with Ellis paper. emphasize cardinality over knapsack
The college application problem can be interpreted as a static variant of the Pandora's Box problem proposed by Weitzman \cite{weitzman1979}. In a Pandora's Box formulation of college application, the student applies to schools one by one, each time paying the application fee and observing her admissions outcome after a certain time delay. The problem is to determine an optimal \emph{stopping rule} for when Pandora should halt her college search and accept the best admissions offer she has on hand. Weitzman showed that the optimal policy is to stop searching when the value of the current best offer exceeds the maximum \emph{reservation price,} a statistic that represents the expected value of applying to a new college.

Arguably, the static model considered in this study is more hostile to students than the Pandora's Box problem. If, for example, an unlucky Pandora is rejected from a safety school at an early round of application, then she can compensate for the unexpected loss by pivoting to a more risk-averse application strategy. By contrast, the decisionmaker in our college application problem must commit at the outset to applying to every school in her application portfolio. The admissions process used in the United States can be viewed as the concatenation of both problems: In the fall, students solve \eqref{headlineproblem} and send out a batch of applications. Then, upon observing their admissions outcomes in March, they use the Pandora strategy to pursue additional offers by applying to schools that offer rolling admissions.

\subsection{Admissions markets}

%Fu \cite{fu2014} considered a problem that is reducible to ours: since application is at cost rather than a constraint, the objective function is separable in the cost and therefore must be optimal over portfolios of the given size. Thus, solving her problem reduces to solving $m$ instances of \eqref{headlineproblem}.

% Merge into above
Some economists have constructed equilibrium models of the college admissions process that feature the student's application decision as a prominent subproblem. Fu \cite{fu2014}, for example, modeled the United States admissions market as a sequential game played by colleges and students. Colleges announce tuition rates, students apply to college, colleges announce admissions decisions and financial aid offers, and lastly, students decide where to enroll. Equilibrium arises when no college can improve its expected utility by modifying its tuition or admissions policy, and no student can improve her expected utility by modifying her application strategy. The ultimate goal of Fu's model was a comparative statics analysis of various hypothetical reforms to the structure of the admissions market; hence, it was adequate to cluster colleges into $m=8$ broad categories. Therefore, though estimating Fu's model required solving a problem similar to \eqref{headlineproblem}, it was a small instance that could be solved by enumerating all possible portfolios. Our study pursues a more general solution.

The literature on admissions markets also includes a vast body of research on the deferred acceptance algorithm, an algorithm for matching students to schools having finite capacities that generalizes the Gale--Shapley algorithm for stable marriage. The algorithm takes students' ordinal preferences over schools and schools' ordinal preferences over students as input, and produces an assignment that possesses a number of desirable properties: It is a stable matching, meaning that no student--school pair is incentivized to deviate from the assignment. And the preference-reporting mechanism is incentive compatible, meaning that no student can improve her outcome by lying about her preferences \cite{galeandshapley1962,roth1982}. In design, deferred acceptance is a centralized assignment algorithm; however, stable assignment also be interpreted as the equilibrium of a decentralized admissions game in which agents have perfect information about one another's preferences \cite{azevedoandleshno2016}.

The utility model implied by stable assignment differs from that of the present study in two key ways: First, under stable assignment, students' preferences are ordinal rather than cardinal, although it is possible to estimate cardinal utility values from stable matchings in certain special cases, such as when colleges' preferences are identical and students' preferences are determined by the multinomial logit choice model \cite{kapur2021}. Second, stable assignment is deterministic: The typical student-proposing deferred acceptance algorithm always produces the unique, student-optimal stable assignment. There exist variants of the deferred acceptance algorithm that introduce randomization in order to break ties in schools' preference lists or optimize for distributional goals such as gender parity, but because the randomization occurs after students submit their preferences, it has no effect on the optimal application strategy, which remains to report one's preferences honestly \cite{ashlagiandnikzad2020,bodohcreed2020}. In the admissions process considered in this study, students face a tougher strategic challenge.

To the best of our knowledge, the first systematic study of the college application problem was undertaken by Kapur in his master's thesis \cite{kapur2022}, which considers \eqref{headlineproblem} as well as the more general case of college application with a knapsack constraint. The present study extends Kapur's results with a further analysis of the accuracy of the heuristics used in the admissions consulting industry and the implications of the nestedness and submodularity properties for student welfare.

\subsection{Outline}

This paper has five sections. In section \ref{sectionModel}, we derive a closed-form expression for the objective function of \eqref{headlineproblem}, show that it is a monotonic submodular function, and state a technical lemma. In section \ref{sectionGreedy}, we prove that the optimal portfolios are nested in the budget $h$, which implies the validity of the greedy algorithm. Further refinement reduces the computation time from $O(h^2 m + m\log m)$ to $O(hm)$. In section \ref{sectionDiscussion}, we examine alternative solution mechanisms used in the admissions consulting industry, their performance relative to the optimal solution, and the implications for student welfare. A brief conclusion follows in section \ref{sectionConclusion}.







\section{Preliminaries} \label{sectionModel}

For the remainder of the paper, unless otherwise noted, we assume with trivial loss of generality that each $f_j \in (0, 1]$ and $< t_1 \leq \cdots \leq t_m$. Unless otherwise noted, we assume $t_0 = 0$, an assumption justified presently. Unless otherwise noted, we assume that $t_0 = 0$, a restriction that we justify presently.

% Lots of overlap here with paper_ellis; edit there first and then import

\subsection{The objective function} \label{sectionObjective}

First we derive a closed-form expression for the objective function of \eqref{headlineproblem}.

We refer to the set $\mathcal{X} \subseteq \mathcal{C}$ of schools to which a student applies as her \emph{application portfolio.} The expected utility the student receives from $\mathcal{X}$ is called its \emph{valuation}. Given an application portfolio, let $p_j(\mathcal{X})$ denote the probability that the student attends school $j$. This occurs if and only if she \emph{applies} to school $j$, is \emph{admitted} to school $j$, and is \emph{rejected} from any school she prefers to $j$; that is, any school with higher index. Hence, for $j= 0\dots m$,
\begin{align}
p_j(\mathcal{X}) &= 
\begin{cases}
\displaystyle f_j  \prod_{\substack{i \in \mathcal{X}: \\ i > j}} (1 - f_{i}), \quad & j \in \{0\}\cup\mathcal{X}\\
0, \quad & \text{otherwise}
\end{cases} 
\end{align}
The following proposition follows by computing $v(\mathcal{X}) = \sum_{j=0}^m  t_j p_j(\mathcal{X})$.
\begin{proposition}[Closed form of portfolio valuation function]
\begin{align}
v(\mathcal{X}) &= \sum_{j=0}^m t_j p_j(\mathcal{X}) = \sum_{j\in\{0\}\cup\mathcal{X}} \Bigl( f_j t_j \prod_{\substack{i \in \mathcal{X}: \\ i > j}} (1 - f_{i}) \Bigr)  \label{closedformportfoliovaluationX}%, \quad \text{or equivalently,}\\
%\qquad v(x) &= t_0 \prod_{j=1}^m (1 - f_{j} x_j) + \sum_{j=1}^m \Bigl( x_j t_j f_j \prod_{j’ = j+1}^m (1 - f_{j’} x_{j’}) \Bigr) \label{closedformportfoliovaluationx}
\end{align}
\end{proposition}
%\begin{proof}Computing $v(\mathcal{X}) = \sum_{j=0}^m  t_j p_j(\mathcal{X})$ yields \eqref{closedformportfoliovaluationX}. Next, because $1 - f_j x_j = 1$ if $x_j = 0$, we may define $p_j$ equivalently as $p_j(x) = x_j  f_j \prod_{j’ = j+1}^m (1 - f_{j’} x_{j’})$ to obtain \eqref{closedformportfoliovaluationx}. 
%\end{proof}

Next, we show that without loss of generality, we may assume that $t_0 = 0$.

\begin{lemma} \label{assumetzerozero}
For some $\gamma \leq t_0$, let $\bar t_j = t_j - \gamma$ for $j = 0 \dots m$. Then $v(\mathcal{X}; \bar t_j) = v(\mathcal{X};  t_j) -  \gamma$ for any $\mathcal{X} \subseteq \mathcal{C}$. 
\end{lemma}
\begin{proof}
By definition, $\sum_{j=0}^m p_j(\mathcal{X}) = \sum_{j \in \{0\}\cup\mathcal{X}} p_j(\mathcal{X}) = 1$. Therefore
\begin{align}
\begin{split}
v(\mathcal{X}; \bar t_j) &= \sum_{j\in \{0\}\cup\mathcal{X}}  \bar t_j p_j(\mathcal{X})
=\sum_{j\in \{0\}\cup\mathcal{X}} (t_j - \gamma) p_j(\mathcal{X}) \\
&=\sum_{j\in \{0\}\cup\mathcal{X}} t_j p_j(\mathcal{X})  - \gamma 
= v(\mathcal{X}; t_j) - \gamma
\end{split} 
\end{align}
which completes the proof.
\end{proof}


\subsection{An elimination technique} \label{eliminationtechniquesection}

Now, we present a variable-elimination technique that will prove useful throughout the paper.\footnote{We thank Yim Seho for pointing out this useful transformation.} Suppose that the student has already resolved to apply to school $k$, and the remainder of her decision consists of determining which \emph{other} schools to apply to. Writing her total application portfolio as $\mathcal{X} = \mathcal{Y} \cup \{k\}$, we devise a function $w(\mathcal{Y})$ that orders portfolios according to how well they ``complement'' the singleton portfolio $\{k\}$. Specifically, the difference between $v(\mathcal{Y} \cup\{k\})$ and $w(\mathcal{Y})$ is the constant $f_k t_k$.

To construct $w(\mathcal{Y})$, let $\tilde t_j$ denote the expected utility the student receives from school $j$ \emph{given} that she has been admitted to school $j$ and applied to school $k$. For $j < k$ (including $j = 0$), this is $\tilde t_j = (1- f_k) t_j + f_k t_k$; for $j > k $, this is $\tilde t_j = t_j$. This means that 
\begin{equation}\label{Vyastildet}
v(\mathcal{Y}\cup\{k\}) = \sum_{j \in \{0\} \cup \mathcal{Y}} \tilde t_j p_j(\mathcal{Y}).\end{equation}
The transformation to $\tilde t$ does not change the order of the $t_j$-values. Therefore, the expression on the right side of \eqref{Vyastildet} is itself a portfolio valuation function. In the corresponding market, $t$ is replaced by $\tilde t$ and $\mathcal{C}$ is replaced by $\mathcal{C}\setminus\{k\}$. To restore our convention that $t_0 = 0$, we obtain $w(\mathcal{Y})$ by taking $\bar t_j = \tilde t_j - \tilde t_0$ for all $j \neq k$ and letting
\begin{equation}  \label{wYvXminusconst}
w(\mathcal{Y})
= \sum_{j \in \{0\} \cup \mathcal{Y}} \bar t_j p_j(\mathcal{Y})
= \sum_{j \in \{0\} \cup \mathcal{Y}} \tilde t_j p_j(\mathcal{Y})- \tilde t_0
= v(\mathcal{Y}\cup\{k\}) -  f_k t_k \end{equation}
where the second equality follows from Lemma \ref{assumetzerozero}. The validity of this transformation is summarized in the following theorem, where we write $v(\mathcal{X}; \bar t)$ instead of $w(\mathcal{Y})$ to emphasize that $w(\mathcal{Y})$ is, in form, a portfolio valuation function. 


\begin{lemma}[Eliminate school $k$] \label{eliminationtheorem}
For $\mathcal{X} \subseteq \mathcal{C} \setminus \{k\}$, $v(\mathcal{X}\cup\{k\}; t)  = v(\mathcal{X}; \bar t) + f_k t_k$, where
\begin{align}\label{howtotransformtj}
\bar t_j = 
\begin{cases}
(1 - f_k) t_j, \quad & t_j \leq t_k \\
t_j - f_k t_k, \quad& t_j > t_k.
\end{cases}
\end{align}
\end{lemma}

\begin{proof}
It is easy to verify that \eqref{howtotransformtj} is the composition of the two transformations (from $t$ to $\tilde t$, and from $\tilde t$ to $\bar t$) discussed above.
\end{proof}

%\noindent The transformation \eqref{howtotransformtj} can be applied iteratively to accommodate the case where the student has already resolved to apply to multiple schools.




\subsection{Submodularity of the objective}

Now, we show that the portfolio valuation function is submodular. This result is primarily of taxonomical interest and may be safely skipped, as our subsequent results do not rely on submodular analysis. 

\begin{definition}[Submodular set function]
Given a ground set $\mathcal{C}$ and function $v : 2^{\mathcal{C}} \mapsto \mathbb{R}$, $v(\mathcal{X})$ is called a \emph{submodular set function} if and only if $v(\mathcal{X}) + v(\mathcal{Y}) \geq v(\mathcal{X}\cup\mathcal{Y}) + v(\mathcal{X}\cap\mathcal{Y})$
for all $\mathcal{X}, \mathcal{Y} \subseteq \mathcal{C}$. Furthermore, if $ v(\mathcal{X}\cup\{k\}) - v(\mathcal{X}) \geq 0$ for all $\mathcal{X} \subset \mathcal{C}$ and $k \in \mathcal{C} \setminus \mathcal{X}$, $v(\mathcal{X})$ is said to be a \emph{nondecreasing} submodular set function.
\end{definition}

\begin{theorem}
The college application portfolio valuation function
$v(\mathcal{X})$ % = \sum_{j\in\mathcal{X}} \Bigl( f_j t_j \prod_{\substack{i \in \mathcal{X}: \\ i > j}} (1 - f_{i}) \Bigr)\]
is a nondecreasing submodular set function.
\end{theorem}

\begin{proof}
It is self-evident that $v(\mathcal{X})$ is nondecreasing. To establish its submodularity, we apply proposition 2.1.iii of \cite{nemhauserandwolsey1978} and show that
\begin{equation}\label{nemhauseriii}
v(\mathcal{X} \cup \{j\}) - v(\mathcal{X}) \geq 
v(\mathcal{X} \cup \{j, k\}) - v(\mathcal{X} \cup \{k\})
\end{equation}
for $\mathcal{X} \subset \mathcal{C}$ and $j \neq k \in \mathcal{C} \setminus \mathcal{X}$. By Lemma \ref{eliminationtheorem}, we can repeatedly eliminate the schools in $\mathcal{X}$ according to \eqref{howtotransformtj} to obtain a portfolio valuation function $w(\mathcal{Y})$ with parameter $\bar t$ such that $w(\mathcal{Y}) = v(\mathcal{X} \cup \mathcal{Y}) + \text{const.}$ for any $\mathcal{Y} \subseteq \mathcal{C} \setminus \mathcal{X}$. Therefore, \eqref{nemhauseriii} is equivalent to
\begin{align}
& w(\{j\}) - w(\varnothing) \geq w(\{j, k\}) - w(\{k\}) \\
\iff \qquad &w(\{j\})  +  w(\{k\})  \geq w(\{j, k\})  \\
\iff \qquad &\operatorname{E}[\,\bar t_j Z_j\,] + \operatorname{E}[\,\bar t_k Z_k\,] 
\geq \operatorname{E}\bigl[\max\{ \bar t_j Z_j, \bar t_k Z_k \} \bigr]
\end{align}
which is plainly true. 
\end{proof}











\section{The greedy algorithm} \label{sectionGreedy}


\subsection{The nestedness property} 
The optimality of the greedy algorithm for the college application problem rests on the fact the the optimal solution possesses a special structure: An optimal portfolio of size $h+1$ includes an optimal portfolio of size $h$ as a subset.

\begin{theorem}[Nestedness of optimal application portfolios] \label{nestedapplication}
There exists a sequence of portfolios $\{\mathcal{X}_h\}_{h=1}^m$ satisfying the nestedness relation

\begin{equation}
\mathcal{X}_1 \subset \mathcal{X}_2\subset \dots \subset \mathcal{X}_m
\end{equation}
such that each $\mathcal{X}_h$ is an optimal application portfolio when the application limit is $h$.
\end{theorem}

\begin{proof}
By induction on $h$. Applying Lemma \ref{assumetzerozero}, we assume that $t_0 = 0$. 

(Base case.) First, we will show that $\mathcal{X}_1 \subset \mathcal{X}_2$. To get a contradiction, suppose that the optima are $\mathcal{X}_1 = \{j\}$ and $\mathcal{X}_2 = \{k, l\}$, where we may assume that $t_k \leq t_l$. Optimality requires that

\begin{equation}v(\mathcal{X}_1 )  = f_j t_j > v(\{k\}) = f_k t_k\end{equation}
and
\begin{align}
\begin{split}
v(\mathcal{X}_2) =  f_k (1- f_l) t_k + f_l t_l &> v(\{j, l\}) \\
& = f_j (1- f_l) t_j + (1- f_j) f_l t_l + f_j f_l \max\{t_j, t_l\} \\
&\geq  f_j (1- f_l) t_j + (1- f_j) f_l t_l + f_j f_l  t_l \\
&= f_j (1- f_l) t_j + f_l t_l  \\
&\geq f_k (1- f_l) t_k + f_l t_l  = v(\mathcal{X}_2)
\end{split}
\end{align}
which is a contradiction. 

(Inductive step.) Assume that $\mathcal{X}_1 \subset \cdots \subset \mathcal{X}_h$, and we will show $\mathcal{X}_h \subset \mathcal{X}_{h+1}$. Let $k = \argmax\{ t_k: k \in \mathcal{X}_{h+1}\}$ and write $\mathcal{X}_{h+1} = \mathcal{Y}_{h} \cup \{k\}$.

Suppose $k \notin \mathcal{X}_h$. To get a contradiction, suppose that $v(\mathcal{Y}_h) < v(\mathcal{X}_h)$ and  $v(\mathcal{X}_{h+1}) > v(\mathcal{X}_h \cup \{k\})$. Then
\begin{align}
\begin{split}
v(\mathcal{X}_{h+1})&= v(\mathcal{Y}_{h} \cup \{k\}) \\
&= (1 - f_k) v(\mathcal{Y}_h) + f_k t_k \\
&\leq (1 - f_k) v(\mathcal{X}_h) + f_k \operatorname{E}\bigl[ \max\{t_k, X_h\}\bigr]\\
&=  v(\mathcal{X}_h\cup \{k\})
\end{split}
\end{align}
is a contradiction.

Now suppose that $k \in \mathcal{X}_h$. We can write $\mathcal{X}_h = \mathcal{Y}_{h-1} \cup \{k\}$, where $ \mathcal{Y}_{h-1}$ is some portfolio of size $h-1$. It suffices to show that $ \mathcal{Y}_{h-1} \subset \mathcal{Y}_h$. By definition, $\mathcal{Y}_{h-1}$ (respectively, $\mathcal{Y}_{h}$) maximizes the function $v(\mathcal{Y}\cup\{k\})$ over portfolios of size $h-1$ (respectively, $h$) that do not include $k$. That is, $\mathcal{Y}_{h-1}$ and $\mathcal{Y}_h$ are the optimal complements to the singleton portfolio $\{k\}$.

Applying Lemma \ref{eliminationtheorem}, we eliminate school $k$, transform the remaining $t_j$-values to $\bar t_j$ according to \eqref{howtotransformtj}, and obtain a function $w(\mathcal{Y}) = v(\mathcal{Y} \cup \{k\}) - f_k t_k$ that grades portfolios $\mathcal{Y} \subseteq \mathcal{C} \setminus \{k\}$ according to how well they complement $\{k\}$. Since $w(\mathcal{Y})$ is itself a portfolio valuation function and $\bar t_0 = 0$, the inductive hypothesis implies that $\mathcal{Y}_{h-1} \subset \mathcal{Y}_h$, which completes the proof.
\end{proof}




\subsection{Quadratic-time solution}

Applying Theorem \ref{nestedapplication} yields an efficient greedy algorithm for the optimal portfolio: Start with the empty set and add schools one at a time, maximizing $v(\mathcal{X}\cup \{k\})$ at each addition. Sorting $t$ is  $O(m \log m)$.  At each of the $h$ iterations, there are $O(m)$ candidates for $k$, and computing $v(\mathcal{X}\cup \{k\})$ is $O(h)$ using \eqref{closedformportfoliovaluationX}; therefore, the time complexity of this algorithm is $O(h^2 m + m \log m)$. 

We reduce the computation time to $O(hm)$ by taking advantage of the transformation from Lemma \ref{eliminationtheorem}. Once school $k$ is added to $\mathcal{X}$, we eliminate it from the set $\mathcal{C}\setminus \mathcal{X}$ of candidates, and update the $t_j$-values of the remaining schools according to \eqref{howtotransformtj}. Now, the \emph{next} school added must be the optimal singleton portfolio in the modified market. But the optimal singleton portfolio consists simply of the school with the highest value of $f_j \bar t_j$. Therefore, by updating the $t_j$-values at each iteration according to \eqref{howtotransformtj}, we eliminate the need to compute $v(\mathcal{X})$ entirely. Moreover, this algorithm does not require the schools to be indexed in ascending order by $t_j$, which removes the $O(m\log m)$ sorting cost.

\begin{algorithm}[h] 
%\DontPrintSemicolon
\caption{Optimal portfolio algorithm for the college application problem.} \label{algorithmforlargeh}
\KwIn{Utility values $t \in(0, \infty)^m$, admissions probabilities $f \in (0, 1]^m$, application limit $h \leq m$.}
$\mathcal{C} \gets \{1 \dots m\}$\;
$\mathtt{X, V} \gets $ empty $h$-arrays\;
\For{$i=1\dots h$}
{
    $k \gets \argmax_{j \in \mathcal{C}}\{f_j t_j\}$\;
    $\mathcal{C} \gets \mathcal{C} \setminus \{k\}$\;
    $\mathtt{X}[i] \gets k$\;
     \lIfElse{$i=1$}{$\mathtt{V}[i] \gets f_k t_k$}
     {$\mathtt{V}[i] \gets \mathtt{V}[i-1] + f_k t_k$}
    \For{$j \in \mathcal{C}$}
	{
	\lIfElse{$t_j \leq t_k$}{$t_j \gets (1 -  f_k) t_j $}{$t_j \gets  t_j -  f_k t_k$}
	}
}
\Return{$\mathtt{X, V}$}
\end{algorithm}


\begin{theorem}[Validity of Algorithm \ref{algorithmforlargeh}] \label{validityofalmaalgorithm}
Algorithm \ref{algorithmforlargeh} produces an optimal application portfolio for the cardinality-constrained college application problem in $O(h m)$ time.
\end{theorem}

\begin{proof}
Optimality follows from the proof of Theorem \ref{nestedapplication}. At each of the $h$ iterations of the main loop, finding the top school costs $O(m)$, and the $t_j$-values of the remaining $O(m)$ schools are each updated in unit time. Therefore, the overall time complexity is $O(h m)$.
\end{proof}

By running Algorithm \ref{algorithmforlargeh} with $h = m$, one can obtain the optimal portfolios for \emph{all} $h$ by taking $\mathcal{X}_h = \{ \mathtt{X}[1], \dots, \mathtt{X}[h]\}$. Thus, a trivial modification of the algorithm yields $O(m^2)$-time solution for Fu (2014)'s variant of the college application problem, stated as 
\[ \text{maximize} \quad v(\mathcal{X}) - c(|\mathcal{X}|)\]
where $c(\cdot)$ is an increasing cost function. By separability, the optimal portfolio in this problem must be optimal over portfolios of the same size. Hence, it suffices to compute
\[h^* = \argmax_{h = 1 \dots m} \bigl\{\mathtt{V}[h] - c(h)\bigr\}\] and return $\mathcal{X}_{h^*}$.  






\section{Discussion: Heuristics and student welfare} \label{sectionDiscussion}

In this section, we discuss a few intuitive heuristics for the college application problem and how their
accuracy compares to the optimal solution. The analysis of these heuristics provides an alternative explanation for findings from the education literature showing that low-income students use a more risk-averse college application strategy than their wealthy peers.

\subsection{The distributional heuristic}

In real-world admissions markets, $f_t$ and $t_j$ correlate inversely because desirable schools attract many applicants. In the admissions consulting industry, schools with high utility and low admissions probability are called reach schools, while schools with low utility and high admissions probability are called safety schools, and those in between are called target schools. According to a common heuristic used by admissions consultants, it is best to apply to a roughly equal mix of target, safety, and reach schools \cite{jeon2015,peck2021}. How does this strategy, which we call the \emph{distributional} heuristic, perform compared to the optimal application strategy?

Note that the distributional heuristic does not specify the boundary between target, safety, and reach schools. However, regardless of how this boundary is specified, we argue that the distributional heuristic is limited by the fact that it does not respond adaptively to the application budget $h$. In a market where $f_j$ and $t_j$ are inversely correlated, the optimal application strategy gradually favors reach schools over safety schools as $h$ increases.

To see why large values of $h$ favor a risky portfolio allocation, consider two schools $i$ and $k$ having $t_i < t_k$ and $f_i t_i = f_k t_k = 1$. Thus, the actuarial value of applying \emph{individually} to either school is identical, and a risk-neutral student is indifferent between the safety school $i$ and the reach school $k$. However, suppose that the student has previously resolved to apply to a target school $j$ having $t_i < t_j < t_k$. It is now optimal to apply to $\mathcal{X} = \{j, k\}$ instead of $\{i, k\}$ as long
as
\begin{align*}
0 &\leq v(\{j, k\}) - v(\{i, j\})\\
&= (1 - f_k ) f_j t_j + f_k t_k - (1 - f_j) f_i t_i - f_j t_j \\
&= \bigl( 1 - \frac{1}{t_k}\bigr) f_j t_j + 1 - (1 - f_j) - f_j t_j \\
&=  - \frac{1}{t_k} f_j t_j + f_j \\
\iff \qquad \frac{f_j t_j}{t_k } &\leq f_j,
\end{align*}
which is \emph{always} true because $t_j < t_k$. This example shows that however one defines a ``target school,'' it is always better to pair a target school with a reach school than with a safety school as long as the choices are equivalent in their actuarial value. By the elimination argument, the same result holds with respect to adding a reach school to a portfolio that is otherwise balanced.


\subsection{The linearization heuristic}

The expected utility associated with applying to school $j$ alone is is simply $\operatorname{E}[t_j Z_j] = f_j t_j$. It is therefore tempting to adopt the following heuristic:
\begin{definition}[Linearization heuristic for the college application problem] \label{naivealgorithm}
Apply to the $h$ schools having the highest expected utility $f_j t_j$.
\end{definition}
\noindent This algorithm's computation time is $O(m)$ using the PICK algorithm of \cite{blumetal1973}. In essence, this heuristic maximizes $\operatorname{E}\left[\,\sum t_j Z_j\, \right]$ as a surrogate for the true objective function $\operatorname{E}\left[\max \{t_j Z_j\} \right]$. Accordingly, we call it the \emph{linearization heuristic.}

The following theorem says that the linearization heuristic is a $(1/h)$-approximation algorithm for the college application problem. 
\begin{theorem} \label{oneoverhoptthm}
When the application limit is $h$, let $\mathcal{X}_h$ denote the optimal portfolio, and $\mathcal{T}_h$ the set of the $h$ schools having the largest values of $f_j t_j$. Then $v(\mathcal{T}_h) / v(\mathcal{X}_h) \geq 1/h$. 
\end{theorem}
\begin{proof}
Because $\mathcal{T}_h$ maximizes the quantity $\operatorname{E}\bigl[ \sum_{j \in \mathcal{T}_h}\{ t_j Z_j \}\bigr]$, we have
\begin{align} \label{oneoverhopt}
\begin{split}
v(\mathcal{X}_h) &= \operatorname{E}\Bigl[ \max_{j \in \mathcal{X}_h}\{ t_j Z_j \}\Bigr] \leq \operatorname{E}\Bigl[ \sum_{j \in \mathcal{X}_h}\{ t_j Z_j \}\Bigr] \leq \operatorname{E}\Bigl[ \sum_{j \in \mathcal{T}_h}\{ t_j Z_j \}\Bigr] \\
&= h  \operatorname{E}\Bigl[ \tfrac{1}{h} \sum_{j \in \mathcal{T}_h}\{ t_j Z_j \}\Bigr]
\leq h  \operatorname{E}\Bigl[ \max_{j \in \mathcal{T}_h}\{ t_j Z_j \}\Bigr]
= h v(\mathcal{T}_h)
\end{split}
\end{align}
where the final inequality follows from the concavity of the $\max\{\}$ operator.
\end{proof}

\noindent The following example establishes the tightness of the approximation factor. 

\begin{example} \label{tightexampleforoneoverhopt}
Pick any $h$ and let $m = 2h$. For a small constant $\varepsilon \in (0, 1)$, define the market as follows.
\begin{center}
\begin{tabular}{r|cccccccc}
$j$   & $1$      & $\cdots$ & $h$   &$h+1$         &  $h+2$ & $\cdots$ &      $m-1$  & $m$            \\ \hline
$f_j$ & $1$     &  $\cdots$ & $1$      & $\varepsilon^{1}$ & $\varepsilon^{2}$ & $\cdots$ & $\varepsilon^{h-1}$ & $\varepsilon^{h}$ \\
$t_j$ & $1$      &  $\cdots$ & $1$      & $\varepsilon^{-1}$ & $\varepsilon^{-2}$ & $\cdots$ & $\varepsilon^{-(h-1)}$ & $\varepsilon^{-h}$
\end{tabular}%
\end{center}
Since all $f_j t_j = 1$, the linearization heuristic can choose $\mathcal{T}_h = \{1, \dots, h\}$, with $v(\mathcal{T}_h) = 1$. But the optimal solution is $\mathcal{X}_h = \{h+1, \dots, m\}$, with
\begin{equation}
v(\mathcal{X}_h) = \sum_{j= h +1}^m \Bigl( f_j t_j \prod_{j' = j+1}^m (1 - f_{j'}) \Bigr) =  \sum_{j= 1}^h  (1 - \varepsilon)^{j} \approx h.
\end{equation}
Thus, as $\varepsilon$ approaches zero, we have $v(\mathcal{T}_h) / v(\mathcal{X}_h) \to 1/h$. (The optimality of $\mathcal{X}_h$ follows from the fact that it achieves the upper bound of Theorem \ref{oneoverhoptthm}.)
\end{example}

Example \ref{tightexampleforoneoverhopt} also illustrates that for large application budget, reach schools offer greater marginal utility than safety schools.



\subsection{Risk management and student welfare}

Further analysis of the optimal application strategy provides a novel explanation of findings from regression studies of college applicant behavior and speculate about policy interventions to improve student equity and welfare. We conjecture that broad policies to decrease the costs of application may yield a greater increase in student equity than interventions that narrowly target high-achieving, low-income students.

Classical portfolio optimization models in the Markowitz \cite{markowitz1952} tradition feature risk aversion as an exogeneous \emph{parameter.} The optimal portfolio is expressed as a proportion of the total budget, and this allocation is constant for budgets of all sizes. Therefore, the value of a rational investor's risk parameter can be determined merely by locating her portfolio allocation along the efficient frontier, even if her total investment budget is unknown. 

In the college application problem, the value $t_0$ of the noncollege option can function as a risk-aversion parameter. However, the analysis of the distributional heuristic reveals that regardless of their underlying risk preferences, students must modulate their risk allocation---that is, the relative share of reach and safety schools in their portfolio---according to the number of schools to which they can afford to apply. Students with a large application budget can tolerate, and indeed benefit, from taking larger risks than those without. This means that in college application, variation in students' risk allocation arises from a combination of variance in underlying preferences and variance in students' application budgets. This property, which we may call \emph{optimal risk elasticity,} is not present in the Markowitz model.

A classic finding of behavioral psychology is that agents with limited resources tend to make more risk-averse decisions than their wealthier peers, especially with respect to education and career choice \cite{hartlaubandschneider2012,vanhuizenandalessie2019}. More concretely, wealthy students tend to apply to more reach schools than low-income students of similar academic ability, a tendency that some education scholars attribute to a lack of teachers or older peers who attended selective universities \cite{hoxbyandavery2012}. But given that low-income students tend to apply to fewer schools overall, optimal risk elasticity provides a rational explanation for the difference between their risk allocation and that of their wealthy peers. Hence, the failure of low-income students to apply to reach schools is not necessarily suboptimal decisionmaking due to risk aversion, but could be \emph{optimal} decisionmaking under a smaller budget. If the latter explanation is correct, then optimal risk elasticity suggests that reducing the costs of college application for all students could improve student equity without the need to overcome psychological factors such as risk aversion and the absence of high-achieving role models. 

Further inspiration for policy interventions to improve student welfare arises in an intrinsic feature of submodular set functions: That they exhibit rapidly diminishing returns. The optimality of the greedy algorithm implies that the gain in student utility from increasing the budget from $h-1$ to $h$ is larger than that from increasing it from $h$ to $h+1$.

\begin{theorem} \label{concavityinh}
For $h = 2 \dots (m-1)$,
\begin{equation}v(\mathcal{X}_h) - v(\mathcal{X}_{h-1}) \geq v(\mathcal{X}_{h+1}) - v(\mathcal{X}_{h}).\end{equation} 
\end{theorem}

\begin{proof}
Applying Theorem \ref{nestedapplication}, we write $\mathcal{X}_h = \mathcal{X}_{h-1} \cup\{j\}$ and $\mathcal{X}_{h+1} = \mathcal{X}_{h-1} \cup\{j, k\}$. By optimality, $v(\mathcal{X}_h) - v(\mathcal{X}_{h-1}) \geq v(\mathcal{X}_{h-1}\cup\{k\}) - v(\mathcal{X}_{h-1})$. By submodularity and nestedness, $v(\mathcal{X}_{h-1}\cup\{k\}) - v(\mathcal{X}_{h-1}) \geq  v(\mathcal{X}_{h}\cup\{k\}) - v(\mathcal{X}_{h}) = v(\mathcal{X}_{h+1}) - v(\mathcal{X}_{h})$.
\end{proof}

\noindent (An elementary proof is provided in \cite{kapur2022}.) It follows that when $\mathcal{X}_h$ is the optimal $h$-portfolio for a given market, $v(\mathcal{X}_h) = t_0 + O(h)$. Example \ref{tightexampleforoneoverhopt}, in which $v(\mathcal{X}_h)$ can be made arbitrarily close to $h$, establishes the tightness of this bound.

The fact that $v(\mathcal{X}_h)$ grows sublinearly in $h$ may provide an economic explanation for the underwhelming efficacy of nudges, such as application fee waivers, designed to encourage talented low-income students to apply to selective universities \cite{gurantzetal2021}. In our model, the marginal value of an additional college application is greatest to students who have both a small application portfolio, and small admissions odds at the schools they are currently applying to---hat is, students whose portolio allocation is skewed toward reach schools. However, fee-waiver interventions have typically targeted high-achieving, low-income students (so-called ``one-offs'' \cite{hoxbyandavery2012,hoxbyandturner2013}) who already have high admissions odds at the schools in their portfolio due a combination of underlying academic talent and the aforementioned tendency of low-income students to favor safety schools. Therefore, our analysis suggests that the students who stand to benefit the most from nudges such as fee waivers are not high-achieving, low-income students, but rather students of middling ability who are prepared for college but unlikely to be admitted to every school. 

%On the other hand, dissuading students from applying to too many reach schools may be a desirable feature in an admissions market inasmuch as it eliminates the administrative costs associated with reading low-quality applications. Our analysis of student utility also can inform the design of centralized admissions markets. In the South Korean college admissions process, students are allowed to apply to only $h = 3$ colleges in the main application cycle, a policy which is thought to promote fairness. Suppose that utility is measured in monetary units, $t_0 = \$100$, and $v(\mathcal{X}_3) = \$130$ for a typical student. Then since $v(\mathcal{X}_4) \leq \$140$, a policy that increments the application limit will be unpopular if its tax cost exceeds $\$10$.

\section{Conclusion} \label{sectionConclusion}


\section{Introduction}

Due to the widespread perception that career earnings depend on college admissions outcomes, the solution of \eqref{headlineproblem} holds monetary value. In the American college consulting industry, students pay private consultants an average of \$200 per hour for assistance in preparing application materials, estimating their admissions odds, and identifying target schools (Sklarow 2018). In Korea, an important revenue stream for admissions consulting firms such as Megastudy (\url{megastudy.net}) and Jinhak (\url{jinhak.com}) is ``mock application'' software that claims to use artificial intelligence to optimize the client's application strategy. 

Most quantitative research on college admissions to date has been descriptive in nature. A standard benchmark for logistic regression techniques, for example, involves estimating students' admissions probabilities from information about their past academic performance (Acharya et al. 2019; Lim 2013). When professional admissions consultants advise their clients on where to apply to college, however, they but typically rely on qualitative heuristics, such as balancing applications between safety and reach schools, rather than optimization models (Peck 2021). (Admissions consultants refer to schools with low utility and high admissions probability as \emph{safety schools}, and those with high utility and low admissions probability as \emph{reach schools.})

In the present study, we formulate the student's college application decision as the following optimization problem:
\begin{align} \label{headlineproblem}
\begin{split}
\text{maximize}\quad & v(\mathcal{X}) =  \operatorname{E}\Bigl[\max\bigr\{t_0,
\max\{t_j Z_j : j \in \mathcal{X}\}\bigr\}\Bigr] \\
\text{subject to}\quad & \mathcal{X} \subseteq \mathcal{C}, ~~|\mathcal{X}| \leq H
\end{split}
\end{align}
Here $\mathcal{C} = \{ 1 \dots m\}$ is an index set; $h > 0$ is the number of students the student can apply to; for $j = 1 \dots m$, $Z_j$ is a random, independent Bernoulli variable with probability $f_j$; and for $j = 0\dots m$, $t_j\geq 0$ is a utility parameter. 

Consider a single prospective student in a college market with $m$ colleges, and let each $t_j$-value indicate the utility she associates with attending school $j$, where her utility is $t_0$ if she does not attend college. Let $h$ denote the number of schools to which the student can afford to apply. Lastly, let $f_j$ denote the student's probability of being admitted to school $j$ if she applies, so that $Z_j$ equals one if she is admitted and zero if not. It is appropriate to assume that the $Z_j$ are statistically independent as long as $f_j$ are probabilities estimated specifically for this student (as opposed to generic acceptance rates). Then the student's objective is to maximize the expected utility associated with the best school she is admitted to. Therefore, her optimal college application strategy is given by the solution $\mathcal{X}$ to the problem above, where $\mathcal{X}$ represents the set of schools to which she applies.

% def greedy, nested
We focus on the special case in which students are constrained in the \emph{number} of schools to which they can apply, and show that the optimal portfolios in this case are nested in the budget constraint, implying the optimality of the greedy algorithm that adds schools one at a time according to how much they increase the objective value. 

\subsection{Literature review}

The objective function is a monotonic submodular function in the sense first described by Nemhauser et al. (1978). They showed that the greedy algorithm is asymptotically $(1 - 1/e)$-optimal for optimizing a monotonic submodular function over a cardinality constraint, and a subsequent result of Nemhauser and Wolsey (1978) proved that this approximation ratio is the best achievable by a polynomial-time algorithm. In the special case of college application, however, we show that the same algorithm is exact. 

% define nestedness
The proof of the optimality of the greedy algorithm rests on the nested structure of the portfolios, a property that is of independent interest. As Rozanov and Tamir (2020) articulate, the knowledge that the optima are nested aids not only in computing the optimal solution, but in the implementation thereof under uncertain information. For example, in the United States, many college applications are due at the beginning of November, and it is typical for students to begin working on their applications during the prior summer because colleges reward students who tailor their essays to the target school. However, students may not know how many schools they can afford to apply to until late October. The nestedness property---or equivalently, the validity of a greedy algorithm---implies that even in the absence of complete budget information, students can begin to carry out the optimal application strategy by writing essays for schools in the order that they enter the optimal portfolio.
% Greedy algorithms produce a \emph{nested} family of solutions parameterized by the budget $H$: If $H \leq H'$, then the greedy solution for budget $H$ is a subset of the greedy solution for budget $H'$.

Fu (2014) considered a problem that is reducible to ours: since application is at cost rather than a constraint, the objective function is separable in the cost and therefore must be optimal over portfolios of the given size. Thus, solving her problem reduces to solving $m$ instances of \eqref{headlineproblem}.

% talk about DA

% pandora's box problem

To the best of our knowledge, the first systematic study of the college application problem was undertaken by Kapur in his master's thesis, which considers \eqref{headlineproblem} as well as the more general case of a knapsack constraint. The present study extends Kapur's results with a further discussion of the application strategies recommended in the admissions consulting industry and how these compare to the optimal strategy.

\subsection{Outline}

This paper has five sections. In section \ref{sectionModel}, we derive a closed-form expression for the objective function of \eqref{headlineproblem} and show that it is a monotonic submodular function. In section \ref{sectionGreedy}, we prove that the optimal portfolios are nested in the budget $h$, which implies the validity of the greedy algorithm. Further refinement reduces the computation time to $O(hm)$. In section \ref{sectionDiscussion}, we examine alternative solution mechanisms used in the admissions consulting industry and their performance relative to the optimal solution. A brief conclusion follows in section \ref{sectionConclusion}.







\section{Preliminaries} \label{sectionModel}

For the remainder of the paper, unless otherwise noted, we assume with trivial loss of generality that each $f_j \in (0, 1]$ and $< t_1 \leq \cdots \leq t_m$. Unless otherwise noted, we assume $t_0 = 0$, an assumption justified presently. Unless otherwise noted, we assume that $t_0 = 0$, a restriction that we justify presently.

% Lots of overlap here with paper_ellis; edit there first and then import

\subsection{The objective function} \label{sectionObjective}

First we derive a closed-form expression for the objective function of \eqref{headlineproblem}.

We refer to the set $\mathcal{X} \subseteq \mathcal{C}$ of schools to which a student applies as her \emph{application portfolio.} The expected utility the student receives from $\mathcal{X}$ is called its \emph{valuation}. Given an application portfolio, let $p_j(\mathcal{X})$ denote the probability that the student attends school $j$. This occurs if and only if she \emph{applies} to school $j$, is \emph{admitted} to school $j$, and is \emph{rejected} from any school she prefers to $j$; that is, any school with higher index. Hence, for $j= 0\dots m$,
\begin{align}
p_j(\mathcal{X}) &= 
\begin{cases}
\displaystyle f_j  \prod_{\substack{i \in \mathcal{X}: \\ i > j}} (1 - f_{i}), \quad & j \in \{0\}\cup\mathcal{X}\\
0, \quad & \text{otherwise}
\end{cases} 
\end{align}
The following proposition follows by computing $v(\mathcal{X}) = \sum_{j=0}^m  t_j p_j(\mathcal{X})$.
\begin{proposition}[Closed form of portfolio valuation function]
\begin{align}
v(\mathcal{X}) &= \sum_{j=0}^m t_j p_j(\mathcal{X}) = \sum_{j\in\{0\}\cup\mathcal{X}} \Bigl( f_j t_j \prod_{\substack{i \in \mathcal{X}: \\ i > j}} (1 - f_{i}) \Bigr)  \label{closedformportfoliovaluationX}%, \quad \text{or equivalently,}\\
%\qquad v(x) &= t_0 \prod_{j=1}^m (1 - f_{j} x_j) + \sum_{j=1}^m \Bigl( x_j t_j f_j \prod_{j’ = j+1}^m (1 - f_{j’} x_{j’}) \Bigr) \label{closedformportfoliovaluationx}
\end{align}
\end{proposition}
%\begin{proof}Computing $v(\mathcal{X}) = \sum_{j=0}^m  t_j p_j(\mathcal{X})$ yields \eqref{closedformportfoliovaluationX}. Next, because $1 - f_j x_j = 1$ if $x_j = 0$, we may define $p_j$ equivalently as $p_j(x) = x_j  f_j \prod_{j’ = j+1}^m (1 - f_{j’} x_{j’})$ to obtain \eqref{closedformportfoliovaluationx}. 
%\end{proof}

Next, we show that without loss of generality, we may assume that $t_0 = 0$.

\begin{lemma} \label{assumetzerozero}
For some $\gamma \leq t_0$, let $\bar t_j = t_j - \gamma$ for $j = 0 \dots m$. Then $v(\mathcal{X}; \bar t_j) = v(\mathcal{X};  t_j) -  \gamma$ for any $\mathcal{X} \subseteq \mathcal{C}$. 
\end{lemma}
\begin{proof}
By definition, $\sum_{j=0}^m p_j(\mathcal{X}) = \sum_{j \in \{0\}\cup\mathcal{X}} p_j(\mathcal{X}) = 1$. Therefore
\begin{align}
\begin{split}
v(\mathcal{X}; \bar t_j) &= \sum_{j\in \{0\}\cup\mathcal{X}}  \bar t_j p_j(\mathcal{X})
=\sum_{j\in \{0\}\cup\mathcal{X}} (t_j - \gamma) p_j(\mathcal{X}) \\
&=\sum_{j\in \{0\}\cup\mathcal{X}} t_j p_j(\mathcal{X})  - \gamma 
= v(\mathcal{X}; t_j) - \gamma
\end{split} 
\end{align}
which completes the proof.
\end{proof}


\subsection{An elimination technique} \label{eliminationtechniquesection}

Now, we present a variable-elimination technique that will prove useful throughout the paper.\footnote{We thank Yim Seho for pointing out this useful transformation.} Suppose that the student has already resolved to apply to school $k$, and the remainder of her decision consists of determining which \emph{other} schools to apply to. Writing her total application portfolio as $\mathcal{X} = \mathcal{Y} \cup \{k\}$, we devise a function $w(\mathcal{Y})$ that orders portfolios according to how well they ``complement'' the singleton portfolio $\{k\}$. Specifically, the difference between $v(\mathcal{Y} \cup\{k\})$ and $w(\mathcal{Y})$ is the constant $f_k t_k$.

To construct $w(\mathcal{Y})$, let $\tilde t_j$ denote the expected utility the student receives from school $j$ \emph{given} that she has been admitted to school $j$ and applied to school $k$. For $j < k$ (including $j = 0$), this is $\tilde t_j = (1- f_k) t_j + f_k t_k$; for $j > k $, this is $\tilde t_j = t_j$. This means that 
\begin{equation}\label{Vyastildet}
v(\mathcal{Y}\cup\{k\}) = \sum_{j \in \{0\} \cup \mathcal{Y}} \tilde t_j p_j(\mathcal{Y}).\end{equation}
The transformation to $\tilde t$ does not change the order of the $t_j$-values. Therefore, the expression on the right side of \eqref{Vyastildet} is itself a portfolio valuation function. In the corresponding market, $t$ is replaced by $\tilde t$ and $\mathcal{C}$ is replaced by $\mathcal{C}\setminus\{k\}$. To restore our convention that $t_0 = 0$, we obtain $w(\mathcal{Y})$ by taking $\bar t_j = \tilde t_j - \tilde t_0$ for all $j \neq k$ and letting
\begin{equation}  \label{wYvXminusconst}
w(\mathcal{Y})
= \sum_{j \in \{0\} \cup \mathcal{Y}} \bar t_j p_j(\mathcal{Y})
= \sum_{j \in \{0\} \cup \mathcal{Y}} \tilde t_j p_j(\mathcal{Y})- \tilde t_0
= v(\mathcal{Y}\cup\{k\}) -  f_k t_k \end{equation}
where the second equality follows from Lemma \ref{assumetzerozero}. The validity of this transformation is summarized in the following theorem, where we write $v(\mathcal{X}; \bar t)$ instead of $w(\mathcal{Y})$ to emphasize that $w(\mathcal{Y})$ is, in form, a portfolio valuation function. 


\begin{lemma}[Eliminate school $k$] \label{eliminationtheorem}
For $\mathcal{X} \subseteq \mathcal{C} \setminus \{k\}$, $v(\mathcal{X}\cup\{k\}; t)  = v(\mathcal{X}; \bar t) + f_k t_k$, where
\begin{align}\label{howtotransformtj}
\bar t_j = 
\begin{cases}
(1 - f_k) t_j, \quad & t_j \leq t_k \\
t_j - f_k t_k, \quad& t_j > t_k.
\end{cases}
\end{align}
\end{lemma}

\begin{proof}
It is easy to verify that \eqref{howtotransformtj} is the composition of the two transformations (from $t$ to $\tilde t$, and from $\tilde t$ to $\bar t$) discussed above.
\end{proof}

%\noindent The transformation \eqref{howtotransformtj} can be applied iteratively to accommodate the case where the student has already resolved to apply to multiple schools.




\subsection{Submodularity of the objective}

Now, we show that the portfolio valuation function is submodular. This result is primarily of taxonomical interest and may be safely skipped, as our subsequent results do not rely on submodular analysis. 

\begin{definition}[Submodular set function]
Given a ground set $\mathcal{C}$ and function $v : 2^{\mathcal{C}} \mapsto \mathbb{R}$, $v(\mathcal{X})$ is called a \emph{submodular set function} if and only if $v(\mathcal{X}) + v(\mathcal{Y}) \geq v(\mathcal{X}\cup\mathcal{Y}) + v(\mathcal{X}\cap\mathcal{Y})$
for all $\mathcal{X}, \mathcal{Y} \subseteq \mathcal{C}$. Furthermore, if $ v(\mathcal{X}\cup\{k\}) - v(\mathcal{X}) \geq 0$ for all $\mathcal{X} \subset \mathcal{C}$ and $k \in \mathcal{C} \setminus \mathcal{X}$, $v(\mathcal{X})$ is said to be a \emph{nondecreasing} submodular set function.
\end{definition}

\begin{theorem}
The college application portfolio valuation function
$v(\mathcal{X})$ % = \sum_{j\in\mathcal{X}} \Bigl( f_j t_j \prod_{\substack{i \in \mathcal{X}: \\ i > j}} (1 - f_{i}) \Bigr)\]
is a nondecreasing submodular set function.
\end{theorem}

\begin{proof}
It is self-evident that $v(\mathcal{X})$ is nondecreasing. To establish its submodularity, we apply proposition 2.1.iii of Nemhauser and Wolsey (1978) and show that
\begin{equation}\label{nemhauseriii}
v(\mathcal{X} \cup \{j\}) - v(\mathcal{X}) \geq 
v(\mathcal{X} \cup \{j, k\}) - v(\mathcal{X} \cup \{k\})
\end{equation}
for $\mathcal{X} \subset \mathcal{C}$ and $j \neq k \in \mathcal{C} \setminus \mathcal{X}$. By Lemma \ref{eliminationtheorem}, we can repeatedly eliminate the schools in $\mathcal{X}$ according to \eqref{howtotransformtj} to obtain a portfolio valuation function $w(\mathcal{Y})$ with parameter $\bar t$ such that $w(\mathcal{Y}) = v(\mathcal{X} \cup \mathcal{Y}) + \text{const.}$ for any $\mathcal{Y} \subseteq \mathcal{C} \setminus \mathcal{X}$. Therefore, \eqref{nemhauseriii} is equivalent to
\begin{align}
& w(\{j\}) - w(\varnothing) \geq w(\{j, k\}) - w(\{k\}) \\
\iff \qquad &w(\{j\})  +  w(\{k\})  \geq w(\{j, k\})  \\
\iff \qquad &\operatorname{E}[\,\bar t_j Z_j\,] + \operatorname{E}[\,\bar t_k Z_k\,] 
\geq \operatorname{E}\bigl[\max\{ \bar t_j Z_j, \bar t_k Z_k \} \bigr]
\end{align}
which is plainly true. 
\end{proof}











\section{The greedy algorithm} \label{sectionGreedy}


\subsection{The nestedness property} 
The optimality of the greedy algorithm for the college application problem rests on the fact the the optimal solution possesses a special structure: An optimal portfolio of size $h+1$ includes an optimal portfolio of size $h$ as a subset.

\begin{theorem}[Nestedness of optimal application portfolios] \label{nestedapplication}
There exists a sequence of portfolios $\{\mathcal{X}_h\}_{h=1}^m$ satisfying the nestedness relation

\begin{equation}
\mathcal{X}_1 \subset \mathcal{X}_2\subset \dots \subset \mathcal{X}_m
\end{equation}
such that each $\mathcal{X}_h$ is an optimal application portfolio when the application limit is $h$.
\end{theorem}

\begin{proof}
By induction on $h$. Applying Lemma \ref{assumetzerozero}, we assume that $t_0 = 0$. 

(Base case.) First, we will show that $\mathcal{X}_1 \subset \mathcal{X}_2$. To get a contradiction, suppose that the optima are $\mathcal{X}_1 = \{j\}$ and $\mathcal{X}_2 = \{k, l\}$, where we may assume that $t_k \leq t_l$. Optimality requires that

\begin{equation}v(\mathcal{X}_1 )  = f_j t_j > v(\{k\}) = f_k t_k\end{equation}
and
\begin{align}
\begin{split}
v(\mathcal{X}_2) =  f_k (1- f_l) t_k + f_l t_l &> v(\{j, l\}) \\
& = f_j (1- f_l) t_j + (1- f_j) f_l t_l + f_j f_l \max\{t_j, t_l\} \\
&\geq  f_j (1- f_l) t_j + (1- f_j) f_l t_l + f_j f_l  t_l \\
&= f_j (1- f_l) t_j + f_l t_l  \\
&\geq f_k (1- f_l) t_k + f_l t_l  = v(\mathcal{X}_2)
\end{split}
\end{align}
which is a contradiction. 

(Inductive step.) Assume that $\mathcal{X}_1 \subset \cdots \subset \mathcal{X}_h$, and we will show $\mathcal{X}_h \subset \mathcal{X}_{h+1}$. Let $k = \argmax\{ t_k: k \in \mathcal{X}_{h+1}\}$ and write $\mathcal{X}_{h+1} = \mathcal{Y}_{h} \cup \{k\}$.

Suppose $k \notin \mathcal{X}_h$. To get a contradiction, suppose that $v(\mathcal{Y}_h) < v(\mathcal{X}_h)$ and  $v(\mathcal{X}_{h+1}) > v(\mathcal{X}_h \cup \{k\})$. Then
\begin{align}
\begin{split}
v(\mathcal{X}_{h+1})&= v(\mathcal{Y}_{h} \cup \{k\}) \\
&= (1 - f_k) v(\mathcal{Y}_h) + f_k t_k \\
&\leq (1 - f_k) v(\mathcal{X}_h) + f_k \operatorname{E}\bigl[ \max\{t_k, X_h\}\bigr]\\
&=  v(\mathcal{X}_h\cup \{k\})
\end{split}
\end{align}
is a contradiction.

Now suppose that $k \in \mathcal{X}_h$. We can write $\mathcal{X}_h = \mathcal{Y}_{h-1} \cup \{k\}$, where $ \mathcal{Y}_{h-1}$ is some portfolio of size $h-1$. It suffices to show that $ \mathcal{Y}_{h-1} \subset \mathcal{Y}_h$. By definition, $\mathcal{Y}_{h-1}$ (respectively, $\mathcal{Y}_{h}$) maximizes the function $v(\mathcal{Y}\cup\{k\})$ over portfolios of size $h-1$ (respectively, $h$) that do not include $k$. That is, $\mathcal{Y}_{h-1}$ and $\mathcal{Y}_h$ are the optimal complements to the singleton portfolio $\{k\}$.

Applying Lemma \ref{eliminationtheorem}, we eliminate school $k$, transform the remaining $t_j$-values to $\bar t_j$ according to \eqref{howtotransformtj}, and obtain a function $w(\mathcal{Y}) = v(\mathcal{Y} \cup \{k\}) - f_k t_k$ that grades portfolios $\mathcal{Y} \subseteq \mathcal{C} \setminus \{k\}$ according to how well they complement $\{k\}$. Since $w(\mathcal{Y})$ is itself a portfolio valuation funtion and $\bar t_0 = 0$, the inductive hypothesis implies that $\mathcal{Y}_{h-1} \subset \mathcal{Y}_h$, which completes the proof.
\end{proof}




\subsection{Quadratic-time solution}

Applying Theorem \ref{nestedapplication} yields an efficient greedy algorithm for the optimal portfolio: Start with the empty set and add schools one at a time, maximizing $v(\mathcal{X}\cup \{k\})$ at each addition. Sorting $t$ is  $O(m \log m)$.  At each of the $h$ iterations, there are $O(m)$ candidates for $k$, and computing $v(\mathcal{X}\cup \{k\})$ is $O(h)$ using \eqref{closedformportfoliovaluationX}; therefore, the time complexity of this algorithm is $O(h^2 m + m \log m)$. 

We reduce the computation time to $O(hm)$ by taking advantage of the transformation from Lemma \ref{eliminationtheorem}. Once school $k$ is added to $\mathcal{X}$, we eliminate it from the set $\mathcal{C}\setminus \mathcal{X}$ of candidates, and update the $t_j$-values of the remaining schools according to \eqref{howtotransformtj}. Now, the \emph{next} school added must be the optimal singleton portfolio in the modified market. But the optimal singleton portfolio consists simply of the school with the highest value of $f_j \bar t_j$. Therefore, by updating the $t_j$-values at each iteration according to \eqref{howtotransformtj}, we eliminate the need to compute $v(\mathcal{X})$ entirely. Moreover, this algorithm does not require the schools to be indexed in ascending order by $t_j$, which removes the $O(m\log m)$ sorting cost.

\begin{algorithm}[h] 
%\DontPrintSemicolon
\caption{Optimal portfolio algorithm for the college application problem.} \label{algorithmforlargeh}
\KwIn{Utility values $t \in(0, \infty)^m$, admissions probabilities $f \in (0, 1]^m$, application limit $h \leq m$.}
$\mathcal{C} \gets \{1 \dots m\}$\;
$\mathtt{X, V} \gets $ empty $h$-arrays\;
\For{$i=1\dots h$}
{
    $k \gets \argmax_{j \in \mathcal{C}}\{f_j t_j\}$\;
    $\mathcal{C} \gets \mathcal{C} \setminus \{k\}$\;
    $\mathtt{X}[i] \gets k$\;
     \lIfElse{$i=1$}{$\mathtt{V}[i] \gets f_k t_k$}
     {$\mathtt{V}[i] \gets \mathtt{V}[i-1] + f_k t_k$}
    \For{$j \in \mathcal{C}$}
	{
	\lIfElse{$t_j \leq t_k$}{$t_j \gets (1 -  f_k) t_j $}{$t_j \gets  t_j -  f_k t_k$}
	}
}
\Return{$\mathtt{X, V}$}
\end{algorithm}


\begin{theorem}[Validity of Algorithm \ref{algorithmforlargeh}] \label{validityofalmaalgorithm}
Algorithm \ref{algorithmforlargeh} produces an optimal application portfolio for the cardinality-constrained college application problem in $O(h m)$ time.
\end{theorem}

\begin{proof}
Optimality follows from the proof of Theorem \ref{nestedapplication}. At each of the $h$ iterations of the main loop, finding the top school costs $O(m)$, and the $t_j$-values of the remaining $O(m)$ schools are each updated in unit time. Therefore, the overall time complexity is $O(h m)$.
\end{proof}

By running the algorithm with $h = m$, one can obtain the optimal portfolio for \emph{all} $h$ by taking $\mathcal{X}_h = \{ \mathtt{X}[1], \dots, \mathtt{X}[h]$. Thus, Fu (2014)'s variant of the college application problem 
\[ \text{maximize} \quad v(\mathcal{X}) - c(|\mathcal{X}|)\]
(where $c(|\mathcal{X}|$ is an arbitrary increasing cost function) can be solved in $O(m^2)$-time by a trivial modification of Algorithm \ref{algorithmforlargeh}: Simply compute
\[h^* = \argmax_{h = 1 \dots m} \bigl\{\mathtt{V}[h] - c(h)\bigr\}\] and return $\mathcal{X}_{h^*}$.  






\section{Discussion} \label{sectionDiscussion}

\subsection{The distributional heuristic}

In real-world admissions markets, attractive schools tend to attract many applicants; therefore, $f_t$ and $t_j$ typically correlate inversely. In the admissions consulting industry, schools with high utility and low admissions probability are called reach schools, while schools with low utility and high admissions probability are called safety schools, and those in the middle are called target schools. According to a common heuristic used by admissions consultants, it is best to apply to a roughly equal mix of target, safety, and reach schools (Jeon 2015, Peck 2021). We call this strategy the \emph{distributional} heuristic.

One weakness of the distributional heuristic is that it does not specify the boundary between target, safety, and reach schools. Moreover, however this boundary is specified, our numerical experiments reveal that as the application budget $h$ increases, the proportion of the optimal portfolio allocated to reach schools increases relative to the proportion allocated to safety schools. That is, there is \emph{no} static risk profile that students can target in order to achieve the optimal portfolio. This observation can be justified analytically. %...

When the application budget is small relative to the size of the admissions market, our algorithms recommend a similar approach (see the example of Subsection \ref{planetsexamplesection}). However, inspecting equation \eqref{howtotransformtj}, which discounts school utility parameters to reflect their marginal value relative to the schools already in the portfolio, reveals that low-utility schools are penalized more harshly than high utility schools in the marginal analysis. Consequentially, as the application budget grows, the optimal portfolio in our model tends to favor reach schools over safety schools. (See Figure \ref{samplemarket} for a clear illustration of this phenomenon.)

To see why large values of $h$ favor a risky portfolio allocation, consider equation \eqref{howtotransformtj}, which discounts school utility parameters to reflect their marginal value relative to the schools already in the portfolio. Suppose that at the previous iteration of the greedy algorithm, school $k$ was chosen, and the present iteration requires a choice between school $i$ with $t_i < t_k$ and school $j$ with $t_j > t_k$. Furthermore, suppose that $f_i t_i = f_j t_j = 1$, meaning that (relatively speaking) school $k$ is a reach school and school $j$ is a safety school. A myopic decisionmaker may view schools $i$ and $j$ as equivalent, but upon applying \eqref{howtotransformtj}, we obtain
\begin{align}
f_i \bar t_i & = f_i t_i - f_k f_i t_i  = 1 - f_k\\
f_j \bar t_j &= f_j t_j - f_j f_k t_k = 1 - f_j f_k t_k \geq 1 - f_j
\end{align}
where the inequality follows from the fact that school $k$ was chosen over school $k$. That is, the safety school's discount is independent of its admissions probability, whereas the reach school's discount is proportional to its (small) admissions probability. A priori, the reach school is more likely to be chosen in this scenario. 

 
%  reveals that low-utility schools are penalized more harshly than high utility schools in the marginal analysis. 
%
%Using our model, we can argue that the distributional heuristic is unnecessarily risk averse, 
%
%A common heuristic used in the admissions consulting industry advises students to apply in roughly equal proportion to reach, target, and safety schools. Reach schools are schools 










\section{Conclusion} \label{sectionConclusion}








\section{References}

%\printbibliography

\parskip 0em
\leftskip 2em
\parindent -2em

\indent\indent Acharya, Mohan S., Asfia Armaan, and Aneeta S. Antony. 2019. ``A Comparison of Regression Models for Prediction of Graduate Admissions.'' In \emph{Second International Conference on Computational Intelligence in Data Science.} \url{https://doi.org/10.1109/ICCIDS.2019.8862140}.

Ashlagi, Itai, and Afshin Nikzad. 2020. ``What Matters in School Choice Tie-Breaking? How Competition Guides Design.'' \emph{Journal of Economic Theory} 190: article no. 105120. \url{https://doi.org/10.1016/j.jet.2020.105120}.

Assad, Arjang. 1985. ``Nested Optimal Policies for Set Functions with Applications to Scheduling.'' \emph{Mathematics of Operations Research} 10 (1): 82--99.

Azevedo, Eduardo and Jacob Leshno. 2016. ``A Supply and Demand Framework for Two-Sided Matching Markets.'' \emph{Journal of Political Economy} 124 (5): 1235--68. \url{https://doi.org/10.1086/687476}. 

Badanidiyuru, Ashwinkumar and Jan Vondrák. 2014. ``Fast Algorithms for Maximizing Submodular Functions.'' In \emph{Proceedings of the 2014 Annual ACM--SIAM Symposium on Discrete Algorithms}, 1497--1514. \url{https://doi.org/10.1137/1.9781611973402.110}.

Balas, Egon and Eitan Zemel. 1980. ``An Algorithm for Large Zero-One Knapsack Problems.'' \emph{Operations Research} 28 (5): 1130--54. \url{https://doi.org/10.1287/opre.28.5.1130}. 

Bezanson, Jeff, Alan Edelman, Stefan Karpinski, and Viral B. Shah. 2017. ``Julia: A Fresh Approach to Numerical Computing.'' \emph{SIAM Review} 59: 65–98. \url{https://doi.org/10.1137/141000671}.

Blum,  Manuel, Robert W. Floyd, Vaughan Pratt, Ronald L. Rivest, and Robert E. Tarjan. 1973. ``Time Bounds for Selection.'' \emph{Journal of Computer and System Sciences} 7 (4): 448--61. \url{https://doi.org/10.1016/S0022-0000(73)80033-9}.

Bodoh-Creed, Aaron. 2020. ``Optimizing for Distributional Goals in School Choice Problems.'' \emph{Management Science} 66 (8): 3657--76. \url{https://doi.org/10.1287/mnsc.2019.3376}.

%Budish, Eric. 2011. ``The Combinatorial Assignment Problem: Approximate Competitive Equilibrium from Equal Incomes.'' \emph{Journal of Political Economy} 119 (6): 1061--1103. \url{https://doi.org/10.1086/664613}. 

% 1 - 1/e opt for the problem in the title
Calinescu, Gruia, Chandra Chekuri, Martin Pál, and Jan Vondrák. 2011. ``Maximizing a Monotone Submodular Function Subject to a Matroid Constraint.'' \emph{SIAM Journal on Computing} 40 (6): 1740--66. \url{https://doi.org/10.1137/080733991}.

Carraway, Robert, Robert Schmidt, and Lawrence Weatherford. 1993. ``An Algorithm for Maximizing Target Achievement in the Stochastic Knapsack Problem with Normal Returns.'' \emph{Naval Research Logistics} 40 (2): 161--73. \url{https://doi.org/10.1002/nav.3220400203}.

Chekuri, Chandra, Jan Vondr\'ak, and Rico Zenklusen. 2014. ``Submodular Function Maximization via the Multilinear Relaxation and Contention Resolution Schemes.'' \emph{SIAM Journal on Computing} 43 (6): 1831--79. \url{https://doi.org/10.1137/110839655}.

Cormen, Thomas, Charles Leiserson, and Ronald Rivest. 1990. \emph{Introduction to Algorithms.} Cambridge, MA: The MIT Press.

Dantzig, George B. 1957. ``Discrete-Variable Extremum Problems.'' \emph{Operations Research} 5 (2): 266--88.

Dean, Brian, Michel Goemans, and Jan Vondr\'ak. 2008. ``Approximating the Stochastic Knapsack Problem: The Benefit of Adaptivity.'' \emph{Mathematics of Operations Research} 33 (4): 945--64. \url{https://doi.org/10.1287/moor.1080.0330}.

Edmonds, Jack. 1971. ``Matroids and the Greedy Algorithm.'' \emph{Mathematical Programming} 1: 127--36. \url{https://doi.org/10.1007/BF01584082}.  


%Fredman, Michael Lawrence and Robert Tarjan. 1987. ``Fibonacci Heaps and Their Uses in Improved Network Optimization Algorithms.'' \emph{Journal of the Association for Computing Machinery} 34 (3): 596--615.

Fu, Chao. 2014. ``Equilibrium Tuition, Applications, Admissions, and Enrollment in the College Market.'' \emph{Journal of Political Economy} 122 (2): 225--81. \url{https://doi.org/10.1086/675503}. 

Gale, David, and Lloyd Shapley. 1962. ``College Admissions and the Stability of Marriage.'' \emph{American Mathematics Monthly} 69 (1): 9--15. \url{https://doi.org/10.2307/2312726}.

Garey, Michael and David Johnson. 1979. \emph{Computers and Intractability: A Guide to the Theory of NP-Completeness.} New York: W. H. Freeman and Company. 

Hartlaub, Vanessa and Thorsten Schneider. 2012. “Educational Choice and Risk Aversion: How Important Is Structural vs. Individual Risk Aversion?” \emph{SOEPpapers on Multidisciplinary Panel Data Research,} no. 433. \url{https://www.diw.de/documents/publikationen/73/diw_01.c.394455.de/diw_sp0433.pdf}.

Jeon, Minhee. 2015. ``[College application strategy] Six chances total\dots divide applications across reach, target, and safety schools'' (in Korean). Jungang Ilbo, Aug. 26. \url{https://www.joongang.co.kr/article/18524069}.

Kahneman, Daniel. 2011. \emph{Thinking, Fast and Slow.} New York: Macmillan.

Kapur, Max. 2021. ``Characterizing Nonatomic Admissions Markets.'' ArXiv, July 3, 2021. \url{https://arxiv.org/abs/2107.01340}. 

Kellerer, Hans, Ulrich Pferschy, and David Pisinger. 2004. \emph{Knapsack Problems.} Berlin: Springer.

%1 - 1/ e for monotonic submodular with knapsack constraints, i.e. Ellis with 가나다
Kulik, Ariel, Hadas Shachnai, and Tami Tamir. 2013. ``Approximations for Monotone and Nonmonotone Submodular Maximization with Knapsack Constraints.'' \emph{Mathematics of Operations Research} 38 (4): 729--39. \url{https://doi.org/10.1287/moor.2013.0592}.

Lim, Daniel Kibum. 2013. ``A Simulation Approach to Predicting College Admissions.'' Master's thesis, University of California Los Angeles. \url{https://escholarship.org/uc/item/8r2695n5}.

Markowitz, Harry. 1952. ``Portfolio Selection.'' \emph{The Journal of Finance} 7 (1): 77--91. \url{https://www.jstor.org/stable/2975974}.

Martello, Silvano and Paolo Toth. 1990. \emph{Knapsack Problems: Algorithms and Computer Implementations.} New York: John Wiley \& Sons. 

Meucci, Attilio. 2005. \emph{Risk and Asset Allocation.} Berlin: Springer-Verlag, 2005. 

% 1 - 1/e inapproximability of submodular maximization
Nemhauser, George and Laurence Wolsey. 1978. ``Best Algorithms for Approximating the Maximum of a Submodular Set Function.'' \emph{Mathematics of Operations Research} 3 (3): 177--88. \url{https://doi.org/10.1287/moor.3.3.177}.

Nemhauser, George, Laurence Wolsey, and Marshall Fisher. 1978. ``An Analysis of Approximations for Maximizing Submodular Set Functions—I.'' \emph{Mathematical Programming} 14: 265--94. 

%Othman, Abraham, Eric Budish, and Tuomas Sandholm. 2010. ``Finding Approximate Competitive Equilibria: Efficient and Fair Course Allocation.'' In \emph{Proceedings of 9th International Conference on Autonomous Agents and Multiagent Systems.} New York: ACM. \url{https://dl.acm.org/doi/abs/10.5555/1838206.1838323}.

Parker, R. Gary and Ronald L. Rardin. 1988. \emph{Discrete Optimization.} San Diego: Academic Press.

Roth, Alvin E. 1982. ``The Economics of Matching: Stability and Incentives.'' \emph{Mathematics of Operations Research} 7 (4): 617--28. \url{https://www.jstor.org/stable/3689483}.

Rozanov, Mark and Arie Tamir. 2020. ``The Nestedness Property of the Convex Ordered Median Location Problem on a Tree.'' \emph{Discrete Optimization} 36: 100581. \url{https://doi.org/10.1016/j.disopt.2020.100581}.

Sklarow, Mark. 2018. \emph{State of the Profession 2018: The 10 Trends Reshaping Independent Educational Consulting.} Technical report, Independent Educational Consultants Association. \url{https://www.iecaonline.com/wp-content/uploads/2020/02/IECA-Current-Trends-2018.pdf}.

Sniedovich, Moshe. 1980. ``Preference Order Stochastic Knapsack Problems: Methodological Issues.'' \emph{The Journal of the Operational Research Society} 31 (11): 1025--32. \url{https://www.jstor.org/stable/2581283}. 

Steinberg, E. and M. S. Parks. 1979. ``A Preference Order Dynamic Program for a Knapsack Problem with Stochastic Rewards.'' \emph{The Journal of the Operational Research Society} 30 (2): 141--47. \url{https://www.jstor.org/stable/3009295}. 

% Another ex of nestedness.
%Tibshirani, Robert. 1996. ``Regression Shrinkage and Selection via the Lasso.'' \emph{Journal of the Royal Statistical Society,. Series B (Methodological)} 58, no. 1: 267--88. \url{https://www.jstor.org/stable/2346178}.

Van Huizen, Thomas and Rob Alessie. 2019. ``Risk Aversion and Job Mobility.’’ \emph{Journal of Economic Behavior \& Organization} 164: 91--106. \url{https://doi.org/10.1016/j.jebo.2019.01.021}.

Vazirani, Vijay. 2001. \emph{Approximation Algorithms.} Berlin: Springer. 

Wolsey, Laurence. 1998. \emph{Integer Programming.} New York: John Wiley \& Sons. 


\end{document}








\ifen \section{Introduction}  \else \section{서론} \fi
\ifen This paper considers the following optimization problem:
\else 본 논문은 다음과 같은 최적화 문제를 고려한다.\fi
%\begin{definition}[Optimal college application portfolio] \label{generalproblemstatement}
\begin{align} \label{headlineproblem}
\begin{split}
\text{maximize}\quad & v(\mathcal{X}) =  \operatorname{E}\Bigl[\max\bigr\{t_0,
\max\{t_j Z_j : j \in \mathcal{X}\}\bigr\}\Bigr] \\
\text{subject to}\quad & \mathcal{X} \subseteq \mathcal{C}, ~~\sum_{j\in \mathcal{X}} g_j \leq H
\end{split}
\end{align}
%\end{definition}
\ifen Here $\mathcal{C} = \{ 1 \dots m\}$ is an index set; $H > 0$ is a budget parameter; for $j = 1 \dots m$, $g_j > 0$ is a cost parameter and $Z_j$ is a random, independent Bernoulli variable with probability $f_j$; and for $j = 0\dots m$, $t_j\geq 0$ is a utility parameter. 
\else 단,  $\mathcal{C} = \{ 1 \dots m\}$은 지표 집합이며 $H$는 예산을 나타내는 모수다. 각 $j = 1 \dots m$에 대해 $g_j > 0$는 비용 모수이며 $Z_j$는 확률 $f_j$를 가지는 서로 독립적인 Bernoulli 변수다. 각 $j = 0 \dots m$에 대해 $t_j\geq 0$는 효용 모수다. \fi

\ifen
We refer to this problem as the \emph{optimal college application} problem, as follows. Consider an admissions market with $m$ colleges. The $j$th college is named $c_j$. Consider a single prospective student in this market, and let each $t_j$-value indicate the utility she associates with attending $c_j$, where her utility is $t_0$ if she does not attend college. Let $g_j$ denote the application fee for $c_j$ and $H$ the student's total budget to spend on application fees. Lastly, let $f_j$ denote the student's probability of being admitted to $c_j$ if she applies, so that $Z_j$ equals one if she is admitted and zero if not. It is appropriate to assume that the $Z_j$ are statistically independent as long as $f_j$ are probabilities estimated specifically for this student (as opposed to generic acceptance rates). Then the student's objective is to maximize the expected utility associated with the best school she is admitted to. Therefore, her optimal college application strategy is given by the solution $\mathcal{X}$ to the problem above, where $\mathcal{X}$ represents the set of schools to which she applies. 
\else
다음 해석에 따라 이를 `대학 입학 지원 최적화 문제'라고 부른다. $m$개의 대학교를 가지는 입학 시장을 고려하자. $j$번째 학교의 이름은 $c_j$다. 어떤 학생이 $c_j$에 입학하게 되면 효용 $t_j$를 얻게 된다고 하자. 단, 어떤 대학에도 진학하지 않는 경우 그 학생의 효용은 $t_0$이다. $c_j$의 지원 비용이 $g_j$이며 학생이 지원에 쓸 수 있는 예산이 $H$라고 하자. 마지막으로, $c_j$에 지원하면 학생이 합격할 확률이 $f_j$이라고 하자. 따라서 합격하면 $Z_j = 1$, 합격 안 하면 $Z_j = 0$이 된다. $f_j$가 (학교의 전체적인 합격률이 아니라) 바로 이 학생의 합격 확률일 때 $Z_j$의 확률적 독립성은 적절한 가정이다. 그러면 학생의 목표는 주어진 예산 안에서 학생이 합격하는 학교들에서 얻는 효용 중 기대 최댓값을 최대화하는 것이다. 위 문제의 최적해가 $\mathcal{X}$일 때, 학생의 최적 대학 지원 전략은 $\mathcal{X}$에  속한 학교로 지원하는 것이다.
\fi

\ifen
The college application problem is not solely of theoretical interest. Due to the widespread perception that career earnings depend on college admissions outcomes, the solution of \eqref{headlineproblem} holds monetary value. In the American college consulting industry, students pay private consultants an average of \$200 per hour for assistance in preparing application materials, estimating their admissions odds, and identifying target schools (Sklarow 2018). In Korea, an important revenue stream for admissions consulting firms such as Megastudy (\url{megastudy.net}) and Jinhak (\url{jinhak.com}) is ``mock application'' software that claims to use artificial intelligence to optimize the client's application strategy. However, while predicting admissions outcomes on a school-by-school basis is a standard benchmark for logistic regression models (Acharya et al. 2019; Lim 2013), we believe our study is the first to focus on the overall application strategy as an optimization problem.

The problem is also conformable to other competitive matching games such as job application. Here, the budget constraint may represent the time needed to complete each application, or a legal limit on the number of applications permitted.
\else
대학 입학 지원 최적화 문제는 이론적인 화제일 뿐만이 아니다. 미래의 소득이 대학 입학 결과로 결정된다는 넓은 인식에 따라, \eqref{headlineproblem}의 최적해는 금전적인 가치를 가지고 있다. 미국 입학 컨설팅 산업에서 대학 지원 서류 작성, 합격 확률 추정, 그리고 지원 학교를 선택하는 데에 자문하는 개인 상담가의 시간당 급료는 평균 200달러다 (Sklarow 2018). 한국에서, 메가스터디(\url{megastudy.net})와 진학(\url{jinhak.com})과 같은 입학 컨설팅 기업의 주된 수입원 중 인공지능을 활용하여 학생의 지원 전략을 최대화한다고 홍보하는 ``모의 지원'' 소프트웨어가 있다. 그러나, 학교 개별 합격 확률을 추정하는 것은 로지스틱 회귀 분석 모형의 익숙한 응용 사례(Acharya et al. 2019; Lim 2013)지만, 전체적인 지원 전략에 집중하여 최적화 문제로 모형화한 것은 본 연구의 새로운 기여로 보인다.

취직 전략과 같은 유사한 경쟁적 매칭 게임에도 위의 문제를 적용할 수 있다. 이때, 예산 제약 조건은 원서를 작성하는 시간이나 지원 개수에 대한 법적인 제한을 나타낼 수 있다.
\fi

\ifen \subsection{Previous literature on college admissions} \else \subsection{입학 시장을 고려한 선행 연구}\fi
\ifen
Some economists have constructed equilibrium models of the college admissions process that feature the student's application decision as a prominent subproblem. Fu (2014), for example, modeled the United States admissions market as a sequential game played by colleges and students. Colleges announce tuition rates, students apply to college, colleges announce admissions decisions and financial aid offers, and lastly, students decide where to enroll. Equilibrium arises when no college can improve its expected utility by modifying its tuition or admissions policy, and no student can improve her expected utility by modifying her application strategy. The ultimate goal of Fu's model was a comparative statics analysis of various hypothetical reforms to the structure of the admissions market; hence, it was adequate to cluster colleges into $m=8$ broad categories. Therefore, though estimating Fu's model required solving a problem similar to \eqref{headlineproblem}, it was a small instance that could be solved by enumerating all possible portfolios.\footnote{Fu's model of student utility uses a cost function in the size of the application portfolio instead of a budget constraint. Since the optimal portfolio in her model must be utility-optimal over portfolios of the same size, solving her student application problem reduces to solving $m$ instances of \eqref{headlineproblem}.} Our study pursues a more general solution.
\else
경제학 분야에서, 입학 과정의 균형 모형에서 학생의 지원 결정이 중요한 부문제로서 등장한 사례가 있다. 예를 들어 Fu (2014)는 미국 대학 시장을 대학교와 학생이 참여하는 순차 게임으로 모형화하였다. 대학은 등록금을 공지하고, 학생이 대학에 지원하고, 대학이 입학과 장학금 결과를 공개한 다음에 학생이 어디에 진학할지 결정하는 모형이며, 균형 조건은 어떤 학교도 등록금이나 입학 기준을 수정하여 효용을 늘릴 수 없고, 어떤 학생도 지원 전략을 바꿔서 더 좋은 결과를 얻을 수 없을 때다. Fu의 궁극적 목표는 입학 시장의 다양한 개혁에 대한 민감도 분석이었으며, 이를 진행하기 위해 학교를 $m=8$개의 범주로 클러스터링할 수 있었다. 따라서 모형을 추정하는 데에 \eqref{headlineproblem}과(와) 비슷한 문제를 풀었으나 모든 포트폴리오를 검증하는 열거법으로 풀 수 있는 작은 인스턴스였다.\footnote{Fu의 모형에서는 예산 제약 조건 대신 지원 포트폴리오의 크기에 대한 비용 함수를 사용한다. 이때 최적 포트폴리오는 같은 크기의 포트폴리오 중에서 효용이 가장 높은 포트폴리오가 되어야 하므로 이 문제를 푸는 것은 \eqref{headlineproblem}을(를) $m$번 푸는 것과 동등하다.} 본 연구는 더 일반적인 $m$에 대한 해법을 추구한다.
\fi

\ifen
The literature on admissions markets also includes a vast body of research on the deferred acceptance algorithm, an algorithm for matching students to schools having finite capacities that generalizes the Gale--Shapley algorithm for stable marriage. The algorithm takes students' ordinal preferences over schools and schools' ordinal preferences over students as input, and produces an assignment that possesses a number of desirable properties: It is a stable matching, meaning that no student--school pair is incentivized to deviate from the assignment. And the preference-reporting mechanism is incentive compatible, meaning that no student can improve her outcome by lying about her preferences (Gale and Shapley 1962; Roth 1982). In design, deferred acceptance is a centralized assignment algorithm; however, stable assignment also be interpreted as the equilibrium of a decentralized admissions game in which agents have perfect information about one another's preferences (Azevedo and Leshno 2016).
\else
입학 시장을 모형화한 문헌은 학교 배정 알고리즘에 관한 연구도 포함하며, 유한한 용량으로 갖춘 학교로 학생을 배정하는 deferred acceptance (DA) 알고리즘은 대표적인 사례다. DA는 Gale--Shapley의 안정 결혼(stable marriage) 알고리즘의 확장으로, 학생과 학교의 서로 대한 선호 순위를 입력하면 이 알고리즘이 출력하는 배정 계획은 몇 가지 바람직한 성질을 가진다. 우선, DA의 출력은 안정 매칭(stable matching)이므로 어떤 학생--학교 쌍도 주어진 배정을 위반할 동기가 없음을 보일 수 있다. 그리고 DA에서 학생이 선호 순위를 신청하는 방법은 유인 부합적(incentive compatible)인 것이며, 학생이 자신의 선호를 부정적으로 나타내어 더 좋은 학교로 배정받을 수 없다고 의미한다 (Gale과 Shapley 1962; Roth 1982). DA는 원래 중앙에서 진행하는 학교 배정 알고리즘으로 설계되었지만, 안정 매칭은 모든 참여자에게 다른 참여자의 선호 순위가 완벽하게 알려진 분산적 입학 게임의 균형으로 해석할 수도 있다 (Azevedo와 Leshno 2016). 
\fi

\ifen 
The utility model implied by stable assignment differs from that of the present study in two key ways: First, under stable assignment, students' preferences are ordinal rather than cardinal.\footnote{It is possible to estimate cardinal utility values from stable matchings in certain special cases, such as when colleges' preferences are identical and students' preferences are determined by the multinomial logit choice model (Kapur 2021).} Second, stable assignment is deterministic: The typical student-proposing deferred acceptance algorithm always produces the unique, student-optimal stable assignment. There exist variants of the deferred acceptance algorithm that introduce randomization in order to break ties in schools' preference lists or optimize for distributional goals such as gender parity, but because the randomization occurs after students submit their preferences, it has no effect on the optimal application strategy, which remains to report one's preferences honestly (Ashlagi and Nikzad 2020; Bodoh-Creed 2020). In the admissions process considered in this study, students face a tougher strategic challenge.
\else
안정 배정 모형에서 말하는 효용의 개념은 본 연구의 효용과 기본적으로 다른 점 2가지 있다. 첫째, 안정 배정에서 학생의 선호 순위는 서수적(ordinal)이며 기수적(cardinal)이 아니다.\footnote{모든 학교의 선호 순위가 동일하고 학생의 선호 순위가 multinomial logit 선택 모형으로 결정되는 경우처럼, 안정 매칭이 주어지면 기수적인 효용 모수를 추정할 수 있는 특수한 경우의 안정 배정이 존재한다 (Kapur 2021).} 둘째, 안정 배정은 결정적인 모형이다: 학생이 학교로 지원하는, 가장 전형적인 DA 알고리즘은 항상 학생에게 최적인 유일한 안정 배정을 출력한다. 확률성을 도입하여 학교 선호 순위의 동점을 처리하거나 성평등 같은 배분적 목적을 최적화하는 DA 알고리즘의 확장이 존재한다. 하지만, 여기서 모든 확률 변수는 학생의 선호 순위를 수집한 다음에 관측되므로 정직한 선호 순위를 신청하는 것이 여전히 학생에게 최적인 지원 전략이다 (Ashlagi와 Nikzad 2020; Bodoh-Creed 2020). 본 논문이 고려하는 입학 과정에서 학생의 지원 전략은 더 어려운 최적화 문제다. 
\fi

\ifen \subsection{Methodological orientation} \else \subsection{방법론적 지향} \fi
\ifen
Our analysis of the college application problem straddles several methodological universes. Its stochastic nature recalls classical portfolio allocation models. However, the knapsack constraint renders the problem NP-complete, and necessitates combinatorial solution techniques. We observe that the objective function is also a submodular set function, though our approximation results suggest that college application is a relatively easy instance of submodular maximization. 
\else
본 연구에서 분석한 대학 지원 문제는 다양한 방법론에 걸쳐 있는 문제다. 확률적인 문제인 만큼 재정학 분야에 뿌리를 가지는 포트폴리오 배분 모형과 비슷하다. 그러나 배낭 제약 조건은 대학 지원 문제를 NP-complete하게 만들며 조합적인 해법이 필요하다. 목적함수는 또한 submodular 집합 함수이지만, 본 연구가 제시하는 근사 해법 결과는 대학 지원 문제가 일반 submodular 함수 최대화 문제의 비교적 쉬운 경우임으로 해석할 수 있다.
\fi

\ifen
In her equilibrium analysis of the American college market, Fu (2014) described college application as a ``nontrivial portfolio problem'' (226). In computational finance, traditional portfolio allocation models weigh the sum of expected profit across all assets against a risk term, yielding a concave maximization problem with linear constraints (Markowitz 1952; Meucci 2005). But college applicants maximize the expected value of their \emph{best} asset: If a student is admitted to her $j$th choice, then she is indifferent as to whether she gets into her $(j+1)$th choice. As a result, student utility is \emph{convex} in the utility associated with individual applications. Risk management is implicit in the college application problem because, in a typical admissions market, college preferability correlates inversely with competitiveness. That is, students negotiate a tradeoff between attractive, selective “reach schools” and less preferable “safety schools” where admission is a safer bet (Jeon 2015). Finally, the combinatorial nature of the college application problem makes it difficult to solve using the gradient-based techniques associated with continuous portfolio optimization.
% moved up
% Fu estimated her equilibrium model (which considers application as a \emph{cost} rather than a constraint) by clustering the schools so that $m=8$, a scale at which enumeration is tractable. 
\else
미국 대학 입학 시장의 균형 분석에서, Fu (2014)는 대학 지원 전략을 ``nontrivial portfolio problem''으로 묘사하였다 (226). 재정학 분야에서, 고전적 포트폴리오 배분 최적화 모형은 전체 자산에 대한 기대 총이익에서 위험회피 항을 뺌으로 선형식으로 제약된 오목 최대화 문제를 이룬다 (Markowitz 1952; Meucci 2005). 그러나 대학 지원자는 가치가 제일 높은 단일 자산의 기대 가치를 최대화하고자 한다. 어떤 학생이 자신이 $j$번째로 선호하는 학교에 합격하면 $(j+1)$번째 학교의 합격 여부는 무관한 상황이 된다. 이는 학생의 효용을 각 지원 발송의 효용에 대해 오목이 아닌 볼록 함수로 만든다. 또한 입학 시장에서는 전형적으로 대학의 효용과 합격 확률이 서로 반비례하므로 대학 지원 문제는 위험 관리를 포함하게 된다. 특히 선호도가 높으며 붙기 어려운 “상향” 지원 학교(reach school)와 선호도가 낮으며 붙기 쉬운 “안정” 지원 학교(safety school) 사이의 균형을 고려해야 한다 (전민희 2015). 마지막, 대학 지원의 조합적인 본질로 인해 연속적인 포트폴리오 최적화 문제에서 흔히 사용하는 기울기 해법으로는 풀기가 어렵다.
% moved up
% Fu는 지원 비용을 제약 조건 대신 목적함수의 한 항으로 모형화했으며, 균형 모형의 모수를 추정하기 위해 $m=8$이 되도록 학교들의 클러스터를 먼저 구성했다. 이는 열거법을 통해 문 쉽게 풀 수 있는 규모이지만 본 연구는 더 일반적인 $m$에 대한 해법을 추구한다.
\fi

\ifen
The integer formulation of the college application problem can be viewed as a kind of binary knapsack problem with a polynomial objective function of degree $m$. Our branch-and-bound and dynamic programming algorithms closely resemble existing algorithms for knapsack problems (Martello and Toth 1990, \S\,2.5--6). In fact, by manipulating the admissions probabilities, the objective function can be made to approximate a linear function of the characteristic vector to an arbitrary degree of accuracy, a fact that we exploit in our NP-completeness proof. Previous research has introduced various forms of stochasticity to the knapsack problem, including variants in which each item's utility takes a known probability distribution (Steinberg and Parks 1979; Carraway et al. 1993) and an online context in which the weight of each item is observed after insertion into the knapsack (Dean et al. 2008). Our problem superficially resembles the preference-order knapsack problem considered by Steinberg and Parks and Carraway et al., but these models lack the college application problem's singular ``maximax'' form. Additionally, unlike those models, we do not attempt to replace the real-valued objective function with a preference order over \emph{outcome distributions,} which introduces technical issues concerning competing notions of stochastic dominance (Sniedovich 1980). We take for granted the student's preferences over \emph{outcomes} (as encoded in the $t_j$-values), and focus instead on an efficient computational approach to the well-defined problem above.
\else
대학 지원 문제의 정수 모형은 $m$차 다항식을 목적함수로 갖춘 일종의 이진 배낭 문제로 볼 수 있다. 본 논문에서 제시하는 분지한계법과 동적 계획 해법은 배낭 문제를 위한 기존 알고리즘과 매우 비슷하다 (Martello와 Toth 1990, \S\,2.5--6). 입학 확률을 적당히 조정하면 특성벡터의 선형 함수를 원래 목적함수로 원하는 만큼의 정확성을 가지고 근사할 수 있으며 NP-completeness 증명에서 이 성질을 활용한다. 배낭 문제에 확률성을 도입한 선행 연구 중, 각 상품의 효용이 정해진 확률 분포로 결정되는 모형(Steinberg와 Parks 1979; Carraway 외 1993), 그리고 배낭에 삽입한 다음에 상품의 무게를 관측할 수 있는 온라인 모형 (Dean 외 2008) 등이 있다. 본 연구가 고려하는 문제는 Steinberg와 Parks 그리고 Carraway 외가 고려한 ``우선순위 배낭 문제''와는 유사성을 가지지만, 우선순위 배낭 문제는 대학 지원 문제의 특색인 `maximax' 형태를 가지지는 않는다. 또한 우선순위 모형과 달리, 본 연구에서 실수값 목적함수를 선호 순위를 정의해야 하는 `결과 분포'로 대체할 필요가 없다. 확률적 우위에 대한 상반된 개념들의 문제를 야기시키기 때문이다 (Sniedovich 1980). 대신 $t_j$-값으로 유도된 학생의 `결과'에 대한 선호 순위를 받아들여 위에서 정의한 것처럼 명확히 정의된 문제를 위한 효율적인 계산법을 지향한다.
\fi

\ifen
We take special interest in the validity of greedy optimization algorithms, such as the algorithm that iteratively adds the school that elicits the greatest increase in the objective function until the budget is exhausted. Greedy algorithms produce a \emph{nested} family of solutions parameterized by the budget $H$: If $H \leq H'$, then the greedy solution for budget $H$ is a subset of the greedy solution for budget $H'$. As Rozanov and Tamir (2020) remark, the knowledge that the optima are nested aids not only in computing the optimal solution, but in the implementation thereof under uncertain information. For example, in the United States, many college applications are due at the beginning of November, and it is typical for students to begin working on their applications during the prior summer because colleges reward students who tailor their essays to the target school. However, students may not know how many schools they can afford to apply to until late October. The nestedness property---or equivalently, the validity of a greedy algorithm---implies that even in the absence of complete budget information, students can begin to carry out the optimal application strategy by writing essays for schools in the order that they enter the optimal portfolio.
\else
본 연구에서 특히 탐욕 해법의 가능성에 관심을 기울인다. 예를 들어 예산이 다 소비될 때까지 목적함수를 가장 많이 증가시키는 학교를 차례대로 추가하는 알고리즘은 일종의 탐욕 해법이다. 탐욕 해법은 예산 $H$로 모수화된 해의 순서를 유도하며 그의 원소들은 `포함 사슬 관계'(nestedness)로 연결된다. 즉 $H \leq H'$일 때, 예산 $H$에 해당하는 탐욕 해는 예산 $H'$에 해당하는 탐욕 해의 부분집합이 된다. Rozanov와 Tamir (2020)가 주장하듯, 최적해가 포함 사슬 관계를 가지면 최적해를 구하는 것뿐 아니라 정보가 불확실한 상황에서 최적해를 구현하는 데에도 유용하다. 가령, 많은 미국 대학의 지원 기한은 11월 초인데 학업 계획서를 학교의 취향에 맞춰서 작성해야 하므로 여름부터 원서를 작성하는 학생이 많다. 그러나 지원 예산은 10월 말까지 모를 수도 있다. 포함 사슬 관계, 또는 탐욕 해법의 타당성은 완전한 예산 정보가 없어도 학교가 최적 포트폴리오에 진입하는 순서대로 원서를 작성하면 최적 전략을 구현해 낼 수 있음을 의미한다.
\fi

\ifen
For certain classes of optimization problems, such as maximizing a submodular set function over a cardinality constraint, a greedy algorithm is known to be a good approximate solution and exact under certain additional assumptions (Nemhauser et al. 1978; Assad 1985). For other problems, notably the binary knapsack problem, the most obvious greedy algorithm can be made to perform arbitrarily poorly (Vazirani 2001). We show results for the college application problem that mirror those for the knapsack problem: When each $g_j = 1$, the optimal portfolios are nested. This special case mirrors the centralized college application process in Korea, where there is no application fee, but students are allowed to apply to only three schools during the main admissions cycle. Unfortunately, the nestedness property does not hold in the general case, nor does the greedy algorithm offer any performance guarantee. Instead, we identify a fully polynomial-time approximation scheme (FPTAS) based on fixed-point arithmetic. 
%Instead, we offer a branch-and-bound routine, a pseudopolynomial-time algorithm that is tractable for typical college market instances, and a fully polynomial-time approximation scheme (FPTAS). We also describe a simulated-annealing algorithm that performs well in test instances. 
\else
탐욕 알고리즘이 좋은 근사해거나 정확한 알고리즘이 되는 최적화 모형들이 알려져 있다. 집합 크기 제약 아래서 submodular한 집합 함수를 최대화하는 문제가 대표적인 예다 (Nemhauser 외 1978; Assad 1985). 반면에 이진 배낭 문제 같은 경우에서는 가장 직관적인 탐욕 알고리즘이 최적과 거리가 먼 해를 출력하는 예를 만들 수 있다 (Vazirani 2001). 대학 지원 문제와 배낭 문제가 서로 매우 유사함을 보인다. 모든 $g_j=1$인 특수한 경우에서, 최적 포트폴리오가 탐욕 알고리즘의 타당성과 동등한 포함 사슬 관계를 만족하는 것을 증명한다. 이 경우는 지원 비용이 없으며 정시 모집 기간에 학교 3개에만 지원할 수 있는 한국 입학 과정과 같다. 그러나 일반적인 경우에서 포함 사슬 관계가 성립하지 않을뿐더러 탐욕 알고리즘이 어떤 근사 계수를 보장할 수 없을 보일 수 있다. 대신 고정소숫점 산술을 활용한 완전 다항 시간 근사 해법(fully polynomial-time approximation scheme, FPTAS)을 제시한다.
%대신 분지한계법, 전형적인 입학 시장 인스턴스에 대해 효율적인 의사 다항 시간 동적 계획, 그리고 완전 다항 시간 근사 해법(fully polynomial-time approximation scheme, FPTAS)을 제시한다. 가상 인스턴스에 성능이 좋은 모의 담금질(simulated annealing) 해법도 서술한다. 
\fi


\ifen
Finally, we remark that the objective function of \eqref{headlineproblem} is a nondecreasing submodular function. However, our research employs more elementary analytical techniques, and our approximation results are tighter than those associated with generic submodular maximization algorithms. For example, a well-known result of Nemhauser et al. (1978) implies that the greedy algorithm is asymptotically $(1 - 1/e)$-optimal for the $g_j = 1$ case of the college application problem, whereas we show that the same algorithm is exact. As for the general problem, an equivalent approximation ratio is achievable when maximizing a submodular set function over a knapsack constraint using a variety of relaxation techniques (Chekuri et al. 2014; Badanidiyuru and Vondrák 2014; Kulik et al. 2013). Indeed, $1 - 1/e$ is the highest approximation coefficient achievable by a polynomial-time algorithm for this problem (Nemhauser and Wolsey 1978). But in the college application problem, the existence of the FPTAS supersedes these results.
\else
마지막으로, \eqref{headlineproblem}의 목적함수는 submodular하며 단조 증가하는 함수다. 그런데, 본 연구는 더 기초적인 분석 기술을 이용하며, 일반적인 submodular 최대화 알고리즘보다 좋은 근사 계수를 얻게 된다. 예를 들어 Nemhauser 외 (1978)의 잘 알려진 결과에 따라, 대학 지원 문제의 $g_j = 1$인 경우에서 탐욕 해법의 점근적 근사 계수가 $(1 - 1/e)$이 되며, 본 논문에서 같은 해법이 정확함을 보인다. 일반적인 문제에 대해서는, 배낭 제약식 위에서 submodular 집합 함수를 최대하는 문제에서, 다양한 완화 기술을 활용하면 거의 동등한 근사 계수를 이뤄 낼 수 있다 (Chekuri et al. 2014; Badanidiyuru and Vondrák 2014; Kulik et al. 2013). 이 문제를 위한 다항 시간 해법은 $1 - 1/e$보다 좋은 근사 계수를 가질 수 없음이 알려져 있다 (Nemhauser와 Wolsey 1978). 그러나 대학 지원 문제에서, FPTAS의 존재성은 그보다 강한 결과다.
\fi






\ifen \section{Homogeneous application costs}  \else \section{동일한 지원 비용} \fi \label{homogappcosts}
\ifen In this section, we focus on the special case in which each $g_j = 1$ and $H$ is a natural number $h \leq m$.   We show that an intuitive heuristic is in fact a $1/h$-approximation algorithm, then derive an exact polynomial-time solution algorithm. 
%This case is similar to the centralized college admissions process in Korea, where there is no application fee, but by law, students are allowed to apply to no more than $h$ schools. (In the Korean case, $m=202$ and $h=3$.)
Applying Lemma \ref{assumetzerozero}, we assume that $t_0 = 0$ unless otherwise noted. Throughout this section, we will call the applicant Alma, and refer to the corresponding optimization problem as Alma's problem. 
\else
본 절에서 모든 $g_j = 1$이며 $H$가 자연수 $h \leq m$인 특수한 경우에 집중한다. 직관적인 휴리스틱 해법이 실제로 $1/h$-근사 해법임을 보인 다음에 정확한 다항 시간 해법을 도출한다. 
%이 특수한 경우는 지원 비용이 없으며 국가 법에 따라 최대 $h$개의 학교에 지원할 수 있는 한국의 중심화된 대학 지원 과정과 비슷하다. (한국 경우,  $m=202$이고 $h=3$이다.) 
기본 정리 \ref{assumetzerozero}을(를) 적용하여 다른 언급이 없으면 $t_0 = 0$임을 가정하자. 본 절 내내 지원자를 `알마'라고 부르고 해당 최적화 문제를 `알마의 문제'라고 부른다.
\fi

\begin{problem}[\ifen Alma’s problem\else 알마의 문제\fi]
\ifen Alma's optimal college application portfolio is given by the solution to the following combinatorial optimization problem:
\else
알마의 최적 대학 지원 포트폴리오는 다음 조합 최적화 문제의 최적해다.
\fi
\begin{align}
\begin{split}
\text{maximize}\quad &  v(\mathcal{X}) = \sum_{j\in
%\{0\}\cup
\mathcal{X}} \Bigl( f_j t_j \prod_{\substack{i \in \mathcal{X}: \\ i > j}} (1 - f_{i}) \Bigr)\\
\text{subject to}\quad & \mathcal{X}\subseteq\mathcal{C}, \quad|\mathcal{X}| \leq h 
\end{split}
\end{align}
\end{problem}
%This problem can be formulated as an integer nonlinear program in $m$ binary variables .. 

\ifen \subsection{Approximation properties of a na\"ive solution}  \else \subsection{나이브 해법의 근사 성질} \fi
\ifen
The expected utility associated with a single school $c_j$ is simply $\operatorname{E}[t_j Z_j] = f_j t_j$. It is therefore tempting to adopt the following strategy, which turns out to be suboptimal.
\else 
단일 학교 $c_j$에 해당하는 기대 효용이 단순히 $\operatorname{E}[t_j Z_j] = f_j t_j$이므로 다음과 같은 알고리즘이 매력적으로 보일 수 있지만 사실은 최적해가 아니다.
\fi
\begin{definition}[\ifen Na\"ive algorithm for Alma’s problem\else 알마의 문제를 위한 나이브 해법\fi] \label{naivealgorithm}
\ifen 
Apply to the $h$ schools having the highest expected utility $f_j t_j$.
\else
기대 효용 $f_j t_j$가 가장 큰 $h$개의 학교로 지원한다.
\fi
\end{definition}
\ifen 
\noindent This algorithm's computation time is $O(m)$ using the PICK algorithm of Blum et al. (1973).

The basic error of the na\"ive algorithm is that it maximizes $\operatorname{E}\left[\,\sum t_j Z_j\, \right]$ instead of $\operatorname{E}\left[\max \{t_j Z_j\} \right]$. The latter is what Alma is truly concerned with, since in the end she can attend only one school. In fact, the na\"ive algorithm is a $(1/h)$-approximation algorithm for Alma’s problem, as expressed in the following theorem.
\else
\noindent Blum 외 (1973)의 PICK 알고리즘을 사용하면 위의 해법의 계산 시간이 $O(m)$이다.

나이브 알고리즘의 기본 실수는 $\operatorname{E}\left[\max \{t_j Z_j\} \right]$ 대신  $\operatorname{E}\left[\,\sum t_j Z_j\,\right]$을 최대화하는 것이다. 결국에는 단 한 학교로만 진학할 수 있으므로 알마의 목적은 후자가 아닌 전자다. 다음 정리는 나이브 해법이 알마의 문제를 위한 $(1/h)$-근사 해법임을 의미한다.
\fi

\begin{theorem}[\ifen Accuracy of the na\"ive algorithm\else 나이브 해법의 정확성\fi] \label{oneoverhopt}
\ifen 
When the application limit is $h$, let $\mathcal{X}_h$ denote the optimal portfolio, and $\mathcal{T}_h$ the set of the $h$ schools having the largest values of $f_j t_j$. Then $v(\mathcal{T}_h) / v(\mathcal{X}_h) \geq 1/h$. 
\else
지원 제한이 $h$일 때, 최적 포트폴리오가 $\mathcal{X}_h$이고 기대 효용 $f_j t_j$가 가장 큰 $h$개의 학교의 집합이 $\mathcal{T}_h$라고 하자. 그러면 $v(\mathcal{T}_h) / v(\mathcal{X}_h) \geq 1/h$이다.
\fi
\end{theorem}
\begin{proof}
\ifen
Because $\mathcal{T}_h$ maximizes the quantity $\operatorname{E}\bigl[ \sum_{j \in \mathcal{T}_h}\{ t_j Z_j \}\bigr]$, we have
\else
$\mathcal{T}_h$는 $\operatorname{E}\bigl[ \sum_{j \in \mathcal{T}_h}\{ t_j Z_j \}\bigr]$를 최대화하므로,
\fi
\begin{align} \label{oneoverhopt}
\begin{split}
v(\mathcal{X}_h) &= \operatorname{E}\Bigl[ \max_{j \in \mathcal{X}_h}\{ t_j Z_j \}\Bigr] \leq \operatorname{E}\Bigl[ \sum_{j \in \mathcal{X}_h}\{ t_j Z_j \}\Bigr] \leq \operatorname{E}\Bigl[ \sum_{j \in \mathcal{T}_h}\{ t_j Z_j \}\Bigr] \\
&= h  \operatorname{E}\Bigl[ \tfrac{1}{h} \sum_{j \in \mathcal{T}_h}\{ t_j Z_j \}\Bigr]
\leq h  \operatorname{E}\Bigl[ \max_{j \in \mathcal{T}_h}\{ t_j Z_j \}\Bigr]
= h v(\mathcal{T}_h).
\end{split}
\end{align}
\ifen 
where the final inequality follows from the concavity of the $\max\{\}$ operator.
\else
단, 마지막 부등호는 $\max\{\}$ 연산의 볼록성에 따라 성립한다.
\fi
\end{proof}
\ifen
\noindent The following example establishes the tightness of the approximation factor. 
\else
\noindent 다음 예는 근사 계수가 타이트(tight)함을 보인다.
\fi

\begin{example} \label{tightexampleforoneoverhopt}
\ifen
Pick any $h$ and let $m = 2h$. For a small constant $\varepsilon \in (0, 1)$, define the market as follows.
\else
임의의 $h$를 택하고 $m = 2h$라고 하자. 작은 상수  $\varepsilon \in (0, 1)$에 대해 시장을 다음처럼 구성한다.
\fi
%\begin{align}
% f &= \Bigl(\,\underbrace{1,  \dots, 1}_{h},\;
%\underbrace{\varepsilon^{1}, \varepsilon^{2}, \dots, \varepsilon^{h-1}, \varepsilon^{h}}_{h}\,\Bigr) \\
%\text{\ifen and\else 그리고\fi}\quad t &= \Bigl(\,\underbrace{1,  \dots, 1}_{h},\;
%\underbrace{\varepsilon^{-1}, \varepsilon^{-2}, \dots, \varepsilon^{-(h-1)}, \varepsilon^{-h}}_{h}\,\Bigr) 
%\end{align}
\begin{center}
\begin{tabular}{r|cccccccc}
$j$   & $1$      & $\cdots$ & $h$   &$h+1$         &  $h+2$ & $\cdots$ &      $m-1$  & $m$            \\ \hline
$f_j$ & $1$     &  $\cdots$ & $1$      & $\varepsilon^{1}$ & $\varepsilon^{2}$ & $\cdots$ & $\varepsilon^{h-1}$ & $\varepsilon^{h}$ \\
$t_j$ & $1$      &  $\cdots$ & $1$      & $\varepsilon^{-1}$ & $\varepsilon^{-2}$ & $\cdots$ & $\varepsilon^{-(h-1)}$ & $\varepsilon^{-h}$
\end{tabular}%
\end{center}
\ifen
Since all $f_j t_j = 1$, the na\"ive algorithm can choose $\mathcal{T}_h = \{1, \dots, h\}$, with $v(\mathcal{T}_h) = 1$. But the optimal solution is $\mathcal{X}_h = \{h+1, \dots, m\}$, with
\else
모든 $f_j t_j = 1$이므로 나이브 해법은 $\mathcal{T}_h = \{1, \dots, h\}$를 출력할 수 있으며 $v(\mathcal{T}_h) = 1$이다. 그러나 최적해는 $\mathcal{X}_h = \{h+1, \dots, m\}$이며 그의 기대 효용은 다음과 같다.
\fi
\begin{equation}
v(\mathcal{X}_h) = \sum_{j= h +1}^m \Bigl( f_j t_j \prod_{j' = j+1}^m (1 - f_{j'}) \Bigr) =  \sum_{j= 1}^h  (1 - \varepsilon)^{j} \approx h\ifen.\fi
\end{equation}
\ifen
Thus, as $\varepsilon$ approaches zero, we have $v(\mathcal{T}_h) / v(\mathcal{X}_h) \to 1/h$. (The optimality of $\mathcal{X}_h$ follows from the fact that it achieves the upper bound of Theorem \ref{oneoverhopt}.)
\else
따라서 $\varepsilon$이 0에 가까워지면서 $v(\mathcal{T}_h) / v(\mathcal{X}_h) \to 1/h$이 된다.  ($\mathcal{X}_h$의 최적성은 정리 \ref{oneoverhopt}이(가) 제시하는 상한을 실천하기 때문에 성립한다.)
\fi
\end{example}



%% Wrong: The Fisher theorem would apply to the greedy, i.e. correct algorithm. 
%\begin{corollary}
%\ifen
%The function $v(\mathcal{X})$ is not submodular. 
%\else
%$v(\mathcal{X})$는 하위 모듈 함수가 아니다.
%\fi
%\end{corollary}
%\begin{proof}
%\ifen
%If $v(\mathcal{X})$ is submodular, then theorem 4.2 of Fisher et al. (1978) implies that the na\"ive algorithm achieves an optimality ratio of $v(\mathcal{T}_h) / v(\mathcal{X}_h)  \geq 1 - \left(\frac{h-1}{h}\right)^h$. Example \ref{tightexampleforoneoverhopt} provides a counterexample.
%\else
%$v(\mathcal{X})$가 하위 모듈 함수라면 Fisher 외(1978)의 정리 4.2는 나이브 해법이 $v(\mathcal{T}_h) / v(\mathcal{X}_h)  \geq 1 - \left(\frac{h-1}{h}\right)^h$ 같은 근사 비율을 이룬다고 의미한다. 예 \ref{tightexampleforoneoverhopt}는 반례다.
%\fi
%\end{proof}



\ifen
Although the na\"ive algorithm is suboptimal, we can still find the optimal solution in $O(hm)$ time, as we will now show.
\else
나이브 해법은 최적 해법이 아니지만 $O(hm)$ 시간에 최적해를 구할 수 있다.
\fi

%\ifen We will use the function $w(\mathcal{Y})$ to grade portfolios $\mathcal{Y} \subseteq \mathcal{C} \setminus \{k\}$ according to how well they complement $\{k\}$. To construct $w(\mathcal{Y})$, let $\tilde t_j$ denote the expected utility Alma receives from school $c_j$ \emph{given} that she has been admitted to $c_j$ and applied to $c_k$. For $j < k$, including $j = 0$, this is $\tilde t_j = (1- f_k) t_j + f_k t_k$; for $j > k $, this is $\tilde t_j = t_j$. This means that 
%\else 함수 $w(\mathcal{Y})$를 이용하여 $\mathcal{Y} \subseteq \mathcal{C} \setminus \{k\}$인 포트폴리오와 $\{k\}$ 사이의 \mbox{`보완성'을} 평가하자. $w(\mathcal{Y})$를 구성하기 위해, 알마가 $c_j$에 합격한 상황에서 $c_k$에 지원했을 때 $c_j$에서 받는 조건부 기대 효용을 $\tilde t_j$라고 하자. $j = 0$을 포함한 $j < k$에 대해 이값은 $\tilde t_j = t_j (1- f_k) + t_k f_k$이며, $j > k $에 대해 $\tilde t_j = t_j$임을 알 수 있다. 따라서 \fi
%\begin{equation}\label{Vyastildet}
%v(\mathcal{Y}\cup\{k\}) = \sum_{j \in \{0\} \cup \mathcal{Y}} \tilde t_j p_j(\mathcal{Y}).\end{equation}
%\ifen The transformation to $\tilde t$ does not change the order of the $t_j$-values. Therefore, the expression on the right side of \eqref{Vyastildet} is itself a portfolio valuation function. In the corresponding market, $t$ is replaced by $\tilde t$ and $\mathcal{C}$ is replaced by $\mathcal{C}\setminus\{k\}$. Now, we obtain $w(\mathcal{Y})$ through one more transformation: Define $\bar t_j = \tilde t_j - \tilde t_0$ so that $t_0 = 0$ and let
%\else $\tilde t$로 변환하면 $t_j$의 순서가 변하지 않기 때문에 \eqref{Vyastildet}의 우변 그 자체가 포트폴리오 가치 함수다. $t$를 $\tilde t$로, $\mathcal{C}$를 $\mathcal{C}\setminus \{k\}$로 대체하면 대응하는 시장이 된다. 이제 또 하나의 변환을 적용한다. $t_0 = 0$이 되도록 $\bar t_j = \tilde t_j - \tilde t_0$으로 정의하고 $w(\mathcal{Y})$를 다음과 같이 얻을 수 있다. \fi
%\begin{equation}  \label{wYvXminusconst}
%w(\mathcal{Y})
%= \sum_{j \in \{0\} \cup \mathcal{Y}} \bar t_j p_j(\mathcal{Y})
%= \sum_{j \in \{0\} \cup \mathcal{Y}} \tilde t_j p_j(\mathcal{Y})- \tilde t_0
%= v(\mathcal{Y}\cup\{k\}) -  f_k t_k \end{equation}
%\ifen where the second equality follows from Lemma \ref{assumetzerozero}. This identity says that the optimal complements to $\{k\}$, given by $\mathcal{Y}_{h-1}$ and $\mathcal{Y}_h$, are themselves optimal portfolios of size $h-1$ and $h$ for the market whose objective function is $w(\mathcal{Y})$. Since $\bar t_0 = 0$ in the latter market, the inductive hypothesis implies that $\mathcal{Y}_{h-1} \subset \mathcal{Y}_h$, which completes the proof.\footnote{We thank Yim Seho for discovering this useful transformation.}
%\else 두 번째 등호는 기본 정리 \ref{assumetzerozero}에 따라 성립한다. 이 등호는, $\{k\}$와 최적으로 보완적인 $\mathcal{Y}_{h-1}$과 $\mathcal{Y}_h$는 목적함수가 $w(\mathcal{Y})$인 시장에서 각각 크기가 $h-1$과 $h$인 최적 포트폴리오임을 의미한다. 새로운 시장에서 $\bar t_0 = 0$이므로 귀납법 가설을 적용하면 $\mathcal{Y}_{h-1} \subset \mathcal{Y}_h$임을 알 수 있다. 따라서 증명이 완성된다.\footnote{이 유용한 변환을 발견한 임세호에게 감사를 표한다.} \fi




\ifen 
It is possible to store $\mathcal{C}$ as a binary max heap rather than an array. The heap is ordered according to the criterion $i \geq j \iff f_i t_i \geq f_j t_j$, and by draining the heap and reheapifying at the end of each iteration, the computation time remains $O(hm)$. However, in our numerical experiments, whose results are reported in Section \ref{numericalexperiments},
we found the array implementation to be much faster, because it is possible to identify the entering school $k$ as the utility parameters are updated, which all but eliminates the cost of the $\argmax\{\}$ operation. 
\else
$\mathcal{C}$는 배열 대신 이진 최대 힙 구현으로 저장하는 방법도 고려한다. 단, 힙의 순서는 $i \geq j \iff f_i t_i \geq f_j t_j$의 조건으로 정의하며, 각 반복 단계의 마지막 단계에서 힙을 비우고 다시 힙화(heapify)하면 $O(hm)$ 계산 시간을 유지할 수 있다. 그러나 (\ref{numericalexperiments}절에서 결과를 제시할) 수리 실험에서 배열 구현으로 구성한 알고리즘이 더 빨랐다. 효용 모수를 수정하면서 $k$를 구함으로 $\argmax\{\}$ 연산의 계산 비용을 거의 제거할 수 있기 때문이다.
\fi


\ifen \subsection{Diminishing returns to application} \else\subsection{지원의 수확 체감} \fi
\ifen The nestedness property implies that Alma's expected utility is a discretely concave function of $h$.
\else 포함 사슬 관계 성질은 알마의 기대 효용이 $h$의 이산 오목 함수임을 의미한다.\fi

\begin{theorem}[\ifen Optimal portfolio valuation concave in $h$\else 최적 포트폴리오 가치의 $h$-오목성\fi] \label{concavityinh}
\ifen For $h = 2 \dots (m-1)$,
\else $h = 2 \dots (m-1)$에 대해,\fi
\begin{equation}v(\mathcal{X}_h) - v(\mathcal{X}_{h-1}) \geq v(\mathcal{X}_{h+1}) - v(\mathcal{X}_{h}).\end{equation} 
\end{theorem}


\begin{proof}
\ifen Applying Theorem \ref{nestedapplication}, we write $\mathcal{X}_h = \mathcal{X}_{h-1} \cup\{j\}$ and $\mathcal{X}_{h+1} = \mathcal{X}_{h-1} \cup\{j, k\}$. By optimality, $v(\mathcal{X}_h) - v(\mathcal{X}_{h-1}) \geq v(\mathcal{X}_{h-1}\cup\{k\}) - v(\mathcal{X}_{h-1})$. By submodularity and nestedness, $v(\mathcal{X}_{h-1}\cup\{k\}) - v(\mathcal{X}_{h-1}) \geq  v(\mathcal{X}_{h}\cup\{k\}) - v(\mathcal{X}_{h}) = v(\mathcal{X}_{h+1}) - v(\mathcal{X}_{h})$.
\else 정리 \ref{nestedapplication}을(를) 적용하면 $ \mathcal{X}_h = \mathcal{X}_{h-1} \cup\{j\}$ 그리고 $\mathcal{X}_{h+1} = \mathcal{X}_{h-1} \cup\{j, k\}$으로 표현할 수 있다. 최적성에 따라, $v(\mathcal{X}_h) - v(\mathcal{X}_{h-1}) \geq v(\mathcal{X}_{h-1}\cup\{k\}) - v(\mathcal{X}_{h-1})$. Submodularity 그리고 포함 사슬 관계 성질에 따라, $v(\mathcal{X}_{h-1}\cup\{k\}) - v(\mathcal{X}_{h-1}) \geq  v(\mathcal{X}_{h}\cup\{k\}) - v(\mathcal{X}_{h}) = v(\mathcal{X}_{h+1}) - v(\mathcal{X}_{h})$. \fi
\end{proof}
\ifen
\noindent (An elementary proof is given in \S\,\ref{elementaryconcavityproof}.) It follows that when $\mathcal{X}_h$ is the optimal $h$-portfolio for a given market, $v(\mathcal{X}_h)$ is $O(h)$. In other words, Alma's problem exhibits diminishing returns to the application budget. Example \ref{tightexampleforoneoverhopt}, in which $v(\mathcal{X}_h)$ can be made arbitrarily close to $h$, establishes the tightness of this bound.
\else
\noindent (기초적인 증명은 \S\,\ref{elementaryconcavityproof}에서 제시한다.) 위 결과는 $\mathcal{X}_h$가 어떤 시장의 최적 $h$-포트폴리오일 때, $v(\mathcal{X}_h)$가 $O(h)$ 함수라고 의미한다. 다시 말해 알마 문제의 지원 예산에는 수확 체감 성질이 있다. 예 \ref{tightexampleforoneoverhopt}에서 $v(\mathcal{X}_h)$를 $h$에 임의로 가깝게 조정할 수 있으므로 이 상한이 타이트함을 알 수 있다.
\fi

%To wrap up, we provide an example showing that if the entries of $Z$ are dependent, then the optimal solution may violate the nestedness property of Theorem \ref{nestedapplication}.
%\begin{example}
%Let $t = (3, 3, 4)$, $Z_1 \sim \operatorname{Bernoulli}(0.5)$, $Z_2 = 1 - Z_1$,  and $Z_3 \sim \operatorname{Bernoulli}(0.5)$. Then it is easy to verify that the unique optimal portfolios are $\mathcal{X}_1 = \{3\}$ and $\mathcal{X}_2 = \{1, 2\}$. 
%\end{example}

\ifen \subsection{A small example} \else \subsection{작은 예} \fi \label{planetsexamplesection}
\ifen 
Let us examine a fictional admissions market consisting of $m=8$ schools. The school data, along with the optimal solutions for each $h \leq m$, appear in Table \ref{planetsdata}.
\else 
$m = 8$개의 학교로 구성된 가상 입학 시장을 고려하자. 학교 자료와 각 $h \leq m$에 대응하는 최적해는 표 \ref{planetsdata}에서 나타난다.
\fi

\ifen
Below are shown the first several iterations of Algorithm \ref{algorithmforlargeh}. The values of $f_j$, $t_j$, and their product are recorded only for the schools remaining in $\mathcal{C}$. $f * t$, where $(f * t)_j = f_j t_j$, denotes the entrywise product of $f$ and $t$. The school added at each iteration, underlined, is the one whose $f_j t_j$-value is greatest.
\else
아래에서 알고리즘 \ref{algorithmforlargeh}의 몇 반복 단계가 나타난다. $f_j$, $t_j$, 그리고 그 곱의 값은 $\mathcal{C}$에 남아 있는 학교에 해당하는 값만 기록한다. $(f * t)_j = f_j t_j$로 정의된 $f * t$는 $f$와 $t$의 원소 당 곱을 의미한다. 각 반복 단계에서 최적해에 추가하는 학교는 $f_j t_j$-값이 가장 큰 학교이며 이를 밑줄로 강조한다.
\fi
\newcommand{\Iteration}{\ifen Iteration~\else 반복 단계~\fi}
\begin{align*}
\text{\Iteration 1:}
&&C &= \{1, 2, 3, 4, 5, 6, 7, 8\} \\
&&f &= \{0.39, 0.33, 0.24, 0.24, 0.05, 0.03, 0.1, 0.12\} \\
&&t &= \{200, 250, 300, 350, 400, 450, 500, 550\} \\
&&f * t &= \{78.0, 82.5, 72.0, \underline{84.0}, 20.0, 13.5, 50.0, 66.0\} 
~~\tag*{\(\implies \mathcal{X}_3 = \{4\} \)}\\
\text{\Iteration 2:}
&&C &= \{1, 2, 3, 5, 6, 7, 8\} \\
&&f &= \{0.39, 0.33, 0.24, 0.05, 0.03, 0.1, 0.12\} \\
&&t &= \{152, 190, 228, 316, 366, 416, 466\} \\
&&f * t &= \{59.28, \underline{62.7}, 54.72, 15.8, 10.98, 41.6, 55.92\} 
~~\tag*{\(\implies \mathcal{X}_3 = \{4, 2\} \)}\\
\text{\Iteration 3:}
&&C &= \{1, 3, 5, 6, 7, 8\} \\
&&f &= \{0.39, 0.24, 0.05, 0.03, 0.1, 0.12\} \\
&&t &= \{101.84, 165.3, 253.3, 303.3, 353.3, 403.3\} \\
&&f * t &= \{39.718, 39.672, 12.665, 9.099, 35.33, \underline{48.396}\} 
~~\tag*{\(\implies \mathcal{X}_3 = \{4, 2, 8\} \)}\\
&&&\cdots\\
\text{\Iteration 8:}
&&C &= \{6\}, f = \{0.03\}, t = \{177.622\}, f * t= \{\underline{5.329}\}
~~\tag*{\(\implies \mathcal{X}_3 = \{4, 2, 8, 1, 7, 3, 5, 6\} \)}
\end{align*}
\ifen 
The output of the algorithm is $\mathtt{X} = [4, 2, 8, 1, 7, 3, 5, 6]$, and the optimal $h$-portfolio consists of its first $h$ entries. The ``priority'' column of Table \ref{planetsdata} shows the inverse permutation of $\mathtt{X}$, which is the minimum value of $h$ for which the school is included in the optimal portfolio. Figure \ref{planetsplot} shows the value of the optimal portfolio as a function of $h$. The concave shape of the plot suggests the result of Theorem \ref{concavityinh}. 
\else
알고리즘의 출력은 $\mathtt{X} = [4, 2, 8, 1, 7, 3, 5, 6]$이며 최적 $h$-포트폴리오는 그의 첫 $h$개의 원소로 이루어진다. 표 \ref{planetsdata}에서 등장하는 ``지원 순위'' 자료는 $\mathtt{X}$의 역 순열(inverse permutation)이며 이는 해당 학교가 최적 포트폴리오에 포함되는 최소한 $h$-값을 의미한다. 그림 \ref{planetsplot}은 최적 포트폴리오의 가치를 $h$의 함수로 나타낸다. 곡선의 오목성은 정리 \ref{concavityinh}의 결과를 시사한다.
\fi




\begin{table}[h!] \centering
\small
\begin{tabular}{r|lcccc}
\ifen\textbf{Index $j$} & \textbf{School $c_j$} & \textbf{Admit prob. $f_j$} & \textbf{Utility $t_j$} & \textbf{Priority} & \textbf{$v(\mathcal{X}_h)$} \\ \hline
\else 
\textbf{지표 $j$} & \textbf{학교 $c_j$} & \textbf{합격 확률 $f_j$} & \textbf{효용 $t_j$}  & \textbf{지원 순위} & \textbf{$v(\mathcal{X}_h)$} \\ \hline  \fi
\\[-.75em]
1 & \ifen Mercury University   \else  수성대  \fi   & 0.39   & 200 & 4   & 230.0   \\
2 & \ifen Venus University     \else  금성대  \fi   & 0.33   & 250 & 2   & 146.7  \\
3 & \ifen Mars University      \else  화성대  \fi   & 0.24   & 300 & 6   & 281.5  \\
4 & \ifen Jupiter University   \else  목성대  \fi   & 0.24   & 350 & 1   & \phantom{0}84.0  \\
5 & \ifen Saturn University    \else  토성대  \fi   & 0.05   & 400 & 7   & 288.8  \\
6 & \ifen Uranus University    \else  천왕성대 \fi   & 0.03   & 450 & 8   & 294.1  \\
7 & \ifen Neptune University   \else  해왕성대 \fi   & 0.10   & 500 & 5   & 257.7  \\
8 & \ifen Pluto College        \else  명왕성대 \fi   & 0.12   & 550 & 3   & 195.1      
\end{tabular}
\caption{\label{planetsdata} \normalsize
\ifen College data and optimal application portfolios for a fictional market with $m=8$ schools. By the nestedness property (Theorem \ref{nestedapplication}), the optimal portfolio when the application limit is $h$ consists of the $h$ schools having priority $h$ or less.
\else
$m=8$개의 학교로 이루어진 가상 입학 시장의 대학교 자료와 최적 지원 포트폴리오. 포함 사슬 관계 성질(정리 \ref{nestedapplication})에 따라, 지원 제한이 $h$일 때 최적 포트폴리오는 지원 순위가 $h$ 이하인 $h$개의 학교로 구성된다.
\fi}
\end{table}


\begin{figure}[h!] 
 \centering
% \includegraphics[width=0.95\textwidth]{./plots/h_v-example.pdf}
  \caption{  \label{planetsplot}
  \ifen Application limit $h$ versus the optimal portfolio valuation $v^* = v(\mathcal{X}_h)$ in a fictional market with $m=8$ schools. The concave shape suggests the result of Theorem \ref{concavityinh}. 
  \else $m=8$개의 학교로 구성된 가상 입학 시상에서, 각 지원 제한 $h$에 해당하는 최적 포트폴리오의 가치 $v^* = v(\mathcal{X}_h)$. 곡선의 오목성은 정리 \ref{concavityinh}의 결과를 시사한다.\fi}
\end{figure}






%\pagebreak
\ifen \section{Numerical experiments} \else \section{계산 실험}\fi\label{numericalexperiments}
\ifen
In this section, we present the results of numerical experiments designed to test the practical efficacy of the algorithms derived above. 
\else
본 절에서 위에서 도출한 알고리즘의 현실성을 탐구하고자 계산 실험의 결과를 제시한다.
\fi

\ifen \subsection{Implementation notes} \else \subsection{알고리즘의 구현}\fi
\ifen
We chose to implement our algorithms in the Julia language (v1.8.0b1) because its system of parametric data types allows the compiler to optimize for differential cases such as when the $t_j$-values are integers or floating-point numbers. Julia also offers convenient macros for parallel computing, which enabled us to solve more and larger problems in the benchmark (Bezanson et al. 2017). We made extensive use of the DataStructures.jl package (v0.18.11, \url{github.com/JuliaCollections}). The code is available at \url{github.com/maxkapur/OptimalApplication.jl}. % Or by opening REPL and typing ]add OptimalApplication
\else
모든 알고리즘을 쥴리아 (Julia, v1.8.0b1) 언어로 구현했다. 쥴리아는 $t_j$-값이 정수 또는 부동소숫점 수와 같은 특수한 경우에 대해 컴파일러가 자동으로 최적화하는 모수적 자료형(parametric data type) 기능이 있어서 유용한 선택이었다. 또한 병렬 연산을 쉽게 관리할 수 있는 매크로를 제공하므로 계산 실험의 규모를 증가할 수 있었다 (Bezanson 외 2017). DataStructures.jl 패키지(v0.18.11, \url{github.com/JuliaCollections})도 활용했다. 실험 코드는  \url{github.com/maxkapur/OptimalApplication}에서 공유한다.
\fi

\ifen
The dynamic programs, namely Algorithms \ref{ellisDP1} and \ref{ellisDP3}, were implemented using recursive functions and dictionary memoization, as described in Cormen et al. (1990, \S\,16.6). Our implementation of Algorithm \ref{ellisDP3} also differed from that described in Subsection \ref{fptashead} in that we represented portfolio valuations in \emph{binary} rather than decimal, with the definitions of $P$ and $\mathcal{V}$ modified accordingly, and instead of fixed-point numbers, we worked in integers by multiplying each element of $\mathcal{V}$ by $2^P$. These modifications yield a substantial performance improvement without changing the fundamental algorithm design or complexity analysis.
\else
동적 계획 해법인 \ref{ellisDP1}과(와) 알고리즘 \ref{ellisDP3}을(를) 재귀 함수와 사전 기록으로 구현했다 (Cormen 외 1990, \S\,16.6). 알고리즘 \ref{ellisDP3}은(는) \ref{fptashead}항에서 설명한 것과 달리, 포트폴리오의 가치를 십진수 대신 이진수로 나타냈으며 $P$와  $\mathcal{V}$의 정의를 적절히 수정했다. 그리고 고정소숫점 대신 $\mathcal{V}$의 모든 원소를 $2^P$으로 곱하여 정수로 변환했다. 이런 변경은 기본 알고리즘 설계와 계산 시간 분석에 영향을 주지 않는데 성능이 많이 개선된다.
\fi


\ifen \subsection{Experimental procedure} \else \subsection{실험 방법}\fi
\ifen
We conducted three numerical experiments. Experiments 1 and 2 evaluate the computation times of our algorithms for Alma's problem and Ellis's problem, respectively. Experiment 3 investigates the empirical accuracy of the simulated-annealing heuristic as compared to an exact algorithm. 
\else
총 3개의 수리 실험을 진행했다. 실험 1과 실험 2에서, 각각 알마의 문제와 엘리스의 문제를 위한 해법의 비교적 계산 시간을 판단하고자 한다. 실험 3에서, 정확한 해법에 비한 모의 담금질 휴리스틱의 현실적 정확도를 탐구한다.
\fi


\def\nmarkets{50}
\ifen
To generate synthetic markets, we drew the $t_j$-values independently from an exponential distribution with a scale parameter of ten and rounded up to the nearest integer. To achieve partial negative correlation between $t_j$ and $f_j$, we then set $f_j = 1 / (t_j + 10\,Q)$, where $Q$ is drawn uniformly from the interval $[0, 1)$. In Experiment 1, which concerns Alma's problem, we set $h = \lfloor m/ 2 \rfloor$. In Experiments 2 and 3, which concern Ellis's problem, each $g_j$ is drawn uniformly from the set $\{5, \dots, 10\}$ and we set $H = \lfloor \frac{1}{2} \sum g_j \rfloor$. This cost distribution resembles real-world college application fees, which are often multiples of \$10 between \$50 and \$100. A typical instance is shown in Figure \ref{samplemarket}.
\else
가산 시장을 생성하기 위해 $t_j$-값을 평균이 10인 지수 분포에서 독립적으로 관측하고 정수로 올림했다. 그다음에, $t_j$와 $f_j$가 서로 부분적으로 반비례하게 만들도록 $f_j = 1 / (t_j + 10\,Q)$로 정의했다. 단, $Q$는 $[0, 1)$에서 균일한 확률로 관측한다. 알마의 문제를 고려하는 실험 1에서, $h = \lfloor m/ 2 \rfloor$으로 처리했다. 엘리스의 문제를 고려하는 실험 2와 실험 3에서, 모든  $g_j$를 $\{5, \dots, 10\}$에서 균일한 확률로 관측했으며 $H = \lfloor \frac{1}{2} \sum g_j \rfloor$으로 처리했다. 실제 대학 지원 비용인 50과 100달러 사이에 10달러의 어떤 배수이므로 합리적인 비용 분포다. 그림 \ref{samplemarket}에서 전형적인 인스턴스가 나타난다.
\fi

\ifen
The experimental variables in Experiments 1 and 2 were the market size $m$, the choice of algorithm, and (for Algorithm \ref{ellisDP3}) the tolerance $\varepsilon$. For each combination of the experimental variables, we generated \nmarkets~markets, and the optimal portfolio was computed three times, with the fastest of the three repetitions recorded as the computation time. We then report the mean and standard deviation across the \nmarkets~markets. Therefore, each cell of each table below represents a statistic over {\the\numexpr 3 * \nmarkets \relax}~computations. Where applicable, we do not count the time required to sort the entries of $t$. As the simulated-annealing heuristic's computation time is negligible, it was excluded from Experiment 2. 
\else
실험 1과 실험 2의 실험 변수로는 시장의 크기 $m$, 해법 선택, 그리고 (알고리즘 \ref{ellisDP3}인 경우) 허용치 $\varepsilon$을 고려했다. 실험 변수의 각 조합에 대해,  \nmarkets 개의 시장을 생성했으며 최적 포트폴리오를 3번 계산 한 것 중 가장 빠른 것으로 계산 시간으로 기록했다. \nmarkets 개의 시장에 대한 평균과 표준편차를 표에 나타낸다. 따라서 아래 표의 각 칸은 {\the\numexpr 3 * \nmarkets \relax}개의 계산에 대한 통계다. $t$를 배열하는 시간은 고려하지 않았다. 그리고 모의 담금질 휴리스틱의 계산 시간이 무시해도 될 정도이므로 실험 2에서 제외했다. 
\fi

\ifen
In Experiment 3, we generated 500 markets with sizes varying uniformly in the logarithm between $m = 2^3$ and $2^{11}$. For each market, we computed a heuristic solution using Algorithm \ref{ellissimann} and an exact solution using Algorithm \ref{ellisDP1} and calculated the optimality ratio. We chose to use $N = 500$ iterations in the simulated-annealing algorithm, and the temperature parameters $T = 1/4$ and $r = 1/16$ were selected by a preliminary grid search.  
\else
실험 3에서, 500개의 시장을 생성했으며, 크기는 대수가 균일하게 분포되도록 $m = 2^3$과 $2^{11}$ 사이에서 무작위로 선택했다. 각 시장에 알고리즘 \ref{ellissimann}을(를) 적용하여 휴리스틱한 해를 구하고 알고리즘 \ref{ellisDP1}을(를) 적용하여 정확한 해를 구한 다음의 최적성 비율을 계산했다. 모의 담금질의 반복 단계 개수는 $N = 500$이며, 격자 탐색을 이용한 예비 실험 결과에 따라 온도 모수를 $T = 1/4$ 그리고 $r = 1/16$로 선정했다.
\fi

\ifen \subsection{Summary of results} \else \subsection{결과 정리}\fi
\ifen
Experiment 1 compared the performance of Algorithm \ref{algorithmforlargeh} for homogeneous-cost markets of various sizes when the set of candidate schools $\mathcal{C}$ is stored as an array and as a binary max heap ordered by the $f_j \bar t_j$-values. The results appear in Table \ref{experiment1results}. Our results indicate that the array implementation is faster. %However, the ratio between the average times of the two implementations stays roughly constant, suggesting that a more effective heap implementation could be competitive in certain classes of problem instances. Overall, the rate of growth is quadratic in $m$, which accords with the $O(hm)$ time complexity result of Theorem \ref{validityofalmaalgorithm}.
\else
실험 1에서, 동일한 지원 비용으로 정의된 크기가 다양한 시장에 대해 알고리즘 \ref{algorithmforlargeh}의 성능을 고려했다. 이를 $\mathcal{C}$를 배열 혹은 $f_j \bar t_j$-값에 따라 배열된 힙 구현으로 구성하는 경우로 나눴으며 실험 결과는 표 \ref{experiment1results}에서 등장한다. 실험 결과에 따라 배열 구현이 더 빠른 해법이었다. %그러나, 두개의 방법 사이에 계산 시간 비율은 거의 상수이므로 개선된 힙 해법이 어떤 문제 인스턴스에 대해 더 빠를 가능성이 있어 보인다. 전체적으로 계산 시간은 $m$의 제곱으로 늘아나며 이는 정리 \ref{validityofalmaalgorithm}에서 말한 $O(hm)$ 시간과 동의한다.
\fi

\ifen
In Experiment 2, we turned to the general problem, and compared the performance of the exact algorithms (Algorithms \ref{ellisbnb} and \ref{ellisDP1}) and the approximation scheme (Algorithm \ref{ellisDP3}) at tolerances 0.5 and 0.05.  The results, which appear in Table \ref{experiment2results}, broadly agree with the time complexity analyses presented above. The branch-and-bound algorithm proved impractical for even medium-sized instances. Overall, we found the exact dynamic program to be the fastest algorithm, while the FPTAS was rather slow, a result that echoes the results of computational studies on knapsack problems (Martello and Toth 1990, \S\,2.10). The strong performance of Algorithm \ref{ellisDP1} is partly attributable to the structure of our synthetic instances, in which application costs are small integers and $H$ is proportional in expectation to $m$, meaning the expected computation time is $O(m^2)$. However, the relative advantage of Algorithm \ref{ellisDP1} may be even more pronounced in real college-application instances because the typical student's application budget accommodates at most a dozen or so schools and is constant in $m$. 
\else
실험 2에서 일반적인 문제를 고려한다. 정확한 해법 (알고리즘 \ref{ellisbnb} 및 알고리즘 \ref{ellisDP1}) 그리고 허용치를 0.5와 0.05로 설정한 근사 해법 (알고리즘 \ref{ellisDP3})의 성능을 비교한다. 실험 결과는 표 \ref{experiment1results}에서 나타나며 위에서 도출한 시간 복잡성 결과와 대략 같다. 정확한 동적계획 해법이 전책적으로 가장 빠르며 FPTAS가 약간 느린 것으로 나온 결과는 배낭 문제 해법을 비교한 기존 연구와 비슷하다 (Martello와 Toth 1990, \S\,2.10). 분지한계법은 크기가 중간인 인스턴스에도 비효율적이었다. 알고리즘 \ref{ellisDP1}의 좋은 실험적 성능은 부분적으로 가상 인스턴스의 구조 덕이다. 가상 인스턴스의 지원 비용이 작은 정수이고 $H$가 $m$과 선형식으로 비례하는 것은 알고리즘의 기대 계산 시간이 $O(m^2)$임을 의미한다. 그러나 전형적인 학생의 지원 예산은 최대한 열 몇 개의 학교에 지원할 수 있게 하며 $m$에 대해 상수이므로, 실제 대학 지원 문제 인스턴스에 적용하면 알고리즘 \ref{ellisDP1}이 더욱 뚜렷한 상대적 우위를 발휘할 가능성이 있다. 
\fi

\ifen
The results of Experiment 3, which evaluated the performance of the simulated-annealing heuristic in our synthetic instances, are plotted in Figure \ref{experiment3results}. In the every instance, the heuristic found a solution within 10 percent of optimality, and in the vast majority of instances, it was within 2 percent of optimality. The heuristic performs most poorly in small markets of size $m \leq 16$, but as exact algorithms are tractable at this scale, this result presents no cause for concern.
\else
실험 3은 가상 인스턴스를 통해 모의 담금질 휴리스틱의 현실 성능을 탐구했으며, 실험 결과는 그림 \ref{experiment3results}에 도시하였다. 모든 인스턴스에서 최적의 10\% 이내에 해를 구했으며, 대다수의 경우는 2\% 이내로 달했다. 휴리스틱의 성능은 $m \leq 16$이 되는 작은 시장에서 가장 낮았으며, 정확한 해법도 효율적으로 풀 수 있는 시장 크기이므로 염려할 이유가 없어 보인다.
\fi





\newcommand{\lastptofcaption}{\ifen
For each value of $m$, \nmarkets~markets were generated, and the computation time was recorded as fastest of three repetitions of the algorithm. The table shows the average time (standard deviation) over the \nmarkets~instances.
\else 각 $m$에 대해 \nmarkets 개의 시장을 생성했으며 알고리즘을 3번 반복하여 그중 최소 계산 시간을 기록했다. 표에서 \nmarkets 개의 인스턴스에 대한 평균 (표준편차) 시간이 나타난다.\fi}



\begin{figure}[h!] 
\centering
%\includegraphics[width=0.85\textwidth]{./plots/samplemarket.pdf}
  \caption{\label{samplemarket}
 \ifen A typical randomly-generated instance with $m=64$ schools and its optimal application portfolio. The application costs $g_j$ were drawn uniformly from $\{5, \dots, 10\}$. The optimal portfolio was computed using Algorithm \ref{ellisDP1}.
 \else $m=64$개의 학교로 구성된 전형적인 무작위로 생성한 인스턴스와 해당 최적 포트폴리오. 지원 비용 $g_j$는 $\{5, \dots, 10\}$에서 균일한 확률로 선정했으며 기호의 넓이와 비례된다. 최적 포트폴리오는 알고리즘 \ref{ellisDP1}(으)로 계산했다. \fi}
\end{figure}


\begin{table}[h!] \centering
\small
\begin{tabular}{r|r@{~}r|r@{~}r}
\ifen 
 \textbf{\begin{tabular}[r]{@{}r@{}}Number of\\schools $m$\end{tabular}}& \multicolumn{2}{c|}{\textbf{\begin{tabular}[c]{@{}c@{}}Algorithm \ref{algorithmforlargeh}\\with array\end{tabular}}}  & \multicolumn{2}{c}{\textbf{\begin{tabular}[c]{@{}c@{}}Algorithm \ref{algorithmforlargeh}\\with heap\end{tabular}}}\\ \hline
 \else
 \textbf{\begin{tabular}[r]{@{}r@{}}학교의\\개수 $m$\end{tabular}}& \multicolumn{2}{c|}{\textbf{\begin{tabular}[c]{@{}c@{}}알고리즘 \ref{algorithmforlargeh}\\(배열 구현)\end{tabular}}}  & \multicolumn{2}{c}{\textbf{\begin{tabular}[c]{@{}c@{}}알고리즘 \ref{algorithmforlargeh}\\(힙 구현)\end{tabular}}}\\ \hline
 \fi
    16 &    0.00 &  (0.00) &    0.01 &    (0.00) \\
    64 &    0.01 &  (0.00) &    0.08 &    (0.02) \\
   256 &    0.06 &  (0.00) &    0.96 &    (0.26) \\
  1024 &    0.82 &  (0.03) &   14.27 &    (2.28) \\
  4096 &   12.87 &  (0.71) &  230.77 &   (18.75) \\
 16384 &  199.48 &  (1.77) & 3999.19 &  (283.48)
\end{tabular}
\caption{\label{experiment1results} \normalsize
\ifen (Experiment 1.) Time in ms to compute an optimal portfolio for an admissions market with homogeneous application costs using Algorithm \ref{algorithmforlargeh} when $\mathcal{C}$ is stored as an array and as a heap. \lastptofcaption~In every case, $h = m/2$. 
\else (실험 1.) 단위: ms. 알고리즘 \ref{algorithmforlargeh}에서 $\mathcal{C}$를 배열 또는 힙 구현으로 저장했을 때, 
동일 지원 비용 입학 시장의 최적 지원 포트폴리오를 계산하는 시간. \lastptofcaption~모든 경우, $h = m/2$. \fi}
\end{table}






\begin{table}[h!] \centering
\small
\begin{tabular}{r|r@{~}r|r@{~}r|r@{~}r|r@{~}r}
\ifen
\textbf{\begin{tabular}[r]{@{}r@{}}Number of\\schools $m$\end{tabular}}&\multicolumn{2}{c|}{\textbf{\begin{tabular}[c]{@{}c@{}}Algorithm \ref{ellisbnb}:\\Branch \& bound\end{tabular}}}  & \multicolumn{2}{c|}{\textbf{\begin{tabular}[c]{@{}c@{}}Algorithm \ref{ellisDP1}:\\Costs DP\end{tabular}}}  &\multicolumn{2}{c|}{\textbf{\begin{tabular}[c]{@{}c@{}}Algorithm \ref{ellisDP3}:\\FPTAS, $\varepsilon= 0.5$\end{tabular}}}  & \multicolumn{2}{c}{\textbf{\begin{tabular}[c]{@{}c@{}}Algorithm \ref{ellisDP3}:\\FPTAS, $\varepsilon= 0.05$\end{tabular}}}   \\ \hline
\else
\textbf{\begin{tabular}[r]{@{}r@{}}학교의\\개수 $m$\end{tabular}}&\multicolumn{2}{c|}{\textbf{\begin{tabular}[c]{@{}c@{}}알고리즘  \ref{ellisbnb}:\\분지한계법\end{tabular}}}  & \multicolumn{2}{c|}{\textbf{\begin{tabular}[c]{@{}c@{}}알고리즘 \ref{ellisDP1}:\\지출액 동적 계획\end{tabular}}}  &\multicolumn{2}{c|}{\textbf{\begin{tabular}[c]{@{}c@{}}알고리즘 \ref{ellisDP3}:\\FPTAS, $\varepsilon= 0.5$\end{tabular}}}  & \multicolumn{2}{c}{\textbf{\begin{tabular}[c]{@{}c@{}}알고리즘 \ref{ellisDP3}:\\FPTAS, $\varepsilon= 0.05$\end{tabular}}}   \\ \hline
\fi
   8 &  0.02 & (0.01) &  0.01 & (0.00) &   0.05 &   (0.01) &     0.16 &    (0.04) \\
  16 &  0.10 & (0.04) &  0.07 & (0.02) &   0.39 &   (0.10) &     2.59 &    (0.62) \\
  32 & 12.43 & (9.88) &  0.29 & (0.05) &   2.15 &   (0.29) &    33.07 &   (10.72) \\
  64 &     — &    (—) &  1.24 & (0.16) &  13.24 &   (2.22) &   339.64 &  (100.38) \\
 128 &     — &    (—) &  6.20 & (0.64) &  79.92 &  (20.18) &  2042.63 &  (749.71) \\
 256 &     — &    (—) & 30.63 & (2.28) & 818.50 & (632.98) & 18949.50 & (3533.88)
\end{tabular}%
\caption{\label{experiment2results} \normalsize
\ifen (Experiment 2.) Time in ms to compute an optimal or $(1 - \varepsilon)$-optimal portfolio for an admissions market with heterogeneous application costs using the three algorithms developed in Section \ref{hetappcosts}.
%branch-and-bound routine (Algorithm \ref{ellisbnb}), exact dynamic program (Algorithm \ref{ellisDP1}), and the fully polynomial-time approximation scheme (Algorithm \ref{ellisDP3}).
The branch-and-bound algorithm is impractical for large markets. \lastptofcaption
\else  %분지한계법 (알고리즘 \ref{ellisbnb}), 정확한 동적 계획 (알고리즘 \ref{ellisDP1}), 그리고 완전 다항 시간 근사 해법 (알고리즘 \ref{ellisDP3})을
(실험 2.) 단위: ms. \ref{hetappcosts}절에서 도출한 3개의 알고리즘을 사용할 때, 다양한 지원 비용으로 갖춘 입학 시장의 최적 또는 $(1- \varepsilon)$-최적 포트폴리오를 계산하는 시간. 분지한계법은 큰 시장에서 비실용적이다. \lastptofcaption \fi}
\end{table}




\begin{figure}[h!] 
\centering
%\includegraphics[width=0.95\textwidth]{./plots/accuracy_simulatedannealing.pdf}
  \caption{\label{experiment3results}
 \ifen (Experiment 3.) Optimality ratio achieved by simulated annealing (Algorithm \ref{ellissimann}) in markets of varying size, with parameters $N = 500$, $T = 1/4$, and $r = 1/16$. Optimal portfolios were computed using Algorithm \ref{ellisDP1}. 
 \else  (실험 3.) 크기가 다양한 시장에 적용한 모의 담금질 (알고리즘 \ref{ellissimann}) 해법이 달성하는 최적성 비율. 모의 담금질의 모수는 $N = 500, T = 1/4, r = 1/16$이며, 최적 포트폴리오는 알고리즘 \ref{ellisDP1}(으)로 계산했다. \fi}
\end{figure}








%\pagebreak
\ifen \section{Conclusion and ideas for future research} \else \section{결론과 향후 연구}\fi\label{conclusion}
\ifen
This study has introduced a novel combinatorial optimization problem that we call the college application problem. It can be viewed as a kind of asset allocation or knapsack problem with a submodular objective. We showed that the special case in which colleges have identical application costs can be solved in polynomial time by a greedy algorithm because the optimal solutions are nested in the budget constraint. The general problem is NP-complete. We provided four solution algorithms. The strongest from a theoretical standpoint is an FPTAS that produces a $(1-\varepsilon)$-approximate solution in $O(m^3 / \varepsilon)$ time. On the other hand, for typical college-application instances, the dynamic program based on application expenditures is both easier to implement and substantially more time efficient. A heuristic simulated-annealing algorithm also exhibited strong empirical performance in our computational study. The algorithms discussed in this paper are summarized in Table \ref{algorithmssummary}.
\else
본 연구는 새로운 조합 최적화 문제를 제안했으며 이를 대학 입학 지원 최적화 문제라고 부른다. Submodular한 목적함수를 가지는 자산 배분 혹은 배낭 문제로 볼 수 있다. 모든 대학의 지원 비용이 동일한 특수한 경우, 탐욕 알고리즘으로 다항 시간에 풀 수 있음을 보였으며 최적해가 예산 제약식에 대한 포함 사슬 관계를 가지기 때문이다. 일반적인 문제는 NP-complete하다. 4가지 해법을 제시했으며 이론적으로 가장 효율적인 알고리즘은 $O(m^3 / \varepsilon)$ 시간에  $(1-\varepsilon)$-근사해를 계산하는 FPTAS다. 그 반면에 전형적인 대학 지원 문제 인스턴스에 대해, 지원 비용 지출액 기반 동적 계획 해법은 실행하기 쉬울뿐더러 계산 시간이 상당히 짧다. 계산 실험에서 모의 담금질 기반 휴리스틱 해법도 좋은 성능을 발휘했다. 표 \ref{algorithmssummary}에서 본 논문이 다운 알고리즘을 정리한다.
\fi

\ifen
\begin{table}[h!] \centering
\small
\begin{tabular}{r|lllll}
\textbf{Algorithm} & \textbf{Reference}  & \textbf{Problem} & \textbf{Restrictions} & \textbf{Exactness}       & \textbf{Computation time} \\ \hline
\xrowht[()]{1.5em}  \begin{tabular}[r]{@{}r@{}}Na\"ive\end{tabular} & Definition \ref{naivealgorithm}                   & \begin{tabular}[l]{@{}l@{}}Homogeneous \\ costs\end{tabular}     & None                  & $(1/h)$-opt.               & $O(m)$                    \\ 
\xrowht[()]{1.5em}  Greedy                & Algorithm \ref{algorithmforlargeh} &  \begin{tabular}[l]{@{}l@{}}Homogeneous \\ costs\end{tabular}    & None                  & Exact                    & $O(hm)$                   \\
\xrowht[()]{1.5em}  \begin{tabular}[r]{@{}r@{}}Branch and\\ bound\end{tabular}   & Algorithm \ref{ellisbnb} & General          & None                  & Exact                    & $O(2^m)$                  \\
\xrowht[()]{1.5em}  Costs DP     &  Algorithm \ref{ellisDP1}& General          & $g_j$ integer         & Exact                    & $O(Hm + m \log m)$        \\
\xrowht[()]{1.5em}  FPTAS      &    Algorithm \ref{ellisDP3} & General          &  None         & $(1 - \varepsilon)$-opt. & $O(m^3 / \varepsilon)$   \\
\xrowht[()]{1.5em}  \begin{tabular}[r]{@{}r@{}}Simulated\\annealing\end{tabular}        &    Algorithm \ref{ellissimann} & General          &  None         & $0$-opt. & $O(Nm)$   
\end{tabular}%
\caption{\label{algorithmssummary} \normalsize
Summary of algorithms discussed in this paper.}
\end{table}
\else
\begin{table}[h!] \centering
\small
\begin{tabular}{r|lllll}
\textbf{알고리즘} & \textbf{논문 내 위치}  & \textbf{문제} & \textbf{제한} & \textbf{정확도}       & \textbf{계산 시간} \\ \hline
\xrowht[()]{1.7em}  \begin{tabular}[r]{@{}r@{}}나이브\end{tabular} & 정의 \ref{naivealgorithm}                   & \begin{tabular}[l]{@{}l@{}}동일한\\ 지원 비용\end{tabular}     & 없음                  & $(1/h)$-근사          & $O(m)$                    \\ 
\xrowht[()]{1.7em}  탐욕 해법                & 알고리즘 \ref{algorithmforlargeh} &  \begin{tabular}[l]{@{}l@{}}동일한\\ 지원 비용\end{tabular}    & 없음                  & 정확                    & $O(hm)$                   \\
\xrowht[()]{1.7em}  \begin{tabular}[r]{@{}r@{}}분지한계법\end{tabular}   & 알고리즘 \ref{ellisbnb} & 일반 문제          & 없음                  & 정확                    & $O(2^m)$                  \\
\xrowht[()]{1.7em}   \begin{tabular}[r]{@{}r@{}}지출액\\ 동적 계획\end{tabular}     &  알고리즘 \ref{ellisDP1}& 일반 문제          & $g_j$ 정수         & 정확                    & $O(Hm + m \log m)$        \\
\xrowht[()]{1.7em}  FPTAS      &    알고리즘 \ref{ellisDP3} & 일반 문제          &  없음    & $(1 - \varepsilon)$-근사 & $O(m^3 / \varepsilon)$   \\
\xrowht[()]{1.7em}  모의 담금질      &    알고리즘  \ref{ellissimann} & 일반 문제          &  없음    & $0$-근사 & $O(Nm)$   
\end{tabular}%
\caption{\label{algorithmssummary} \normalsize
본 논문에서 다룬 해법의 요약.}
\end{table}
\fi

\ifen 
Three extensions of this problem appear amenable to future study.
\else
이 문제를 확장할 수 있는 향후 연구 방향 3가지를 제안할 수 있다. 
\fi


\ifen \subsection{Explicit treatment of risk aversion}\else \subsection{위험 회피를 모수화한 모형}\fi
\ifen
The familiar Markowitz portfolio optimization model includes an explicit risk-aversion term, whereas our model has settled for an implicit treatment of risk, namely the tradeoff between selective schools and safety schools inherent in the maximax objective function. However, it is possible to augment the objective function $v(\mathcal{X}) = \operatorname{E}[X]$ to incorporate a variance penalty $\beta \geq 0$ as follows:
\else
익숙한 Markowitz 포트폴리오 최적화 모형에서는 명시적인 위험 회피 항이 등장하며, 본 모형은 maximax 목적함수에 내포된 경쟁적인 학교와 안정 지원 학교 사이의 균형에 의한 위험 관리 요소만 고려했다. 그러나 기본 목적함수인 $v(\mathcal{X}) = \operatorname{E}[X]$에 표준편차 페널티 $\beta \geq 0$를 다음처럼 도입할 수 있다:
\fi
\begin{align}
\begin{split}
v_{\mathrm{\beta}}(\mathcal{X}) &=  \operatorname{E}[X] - \beta \operatorname{Var}(X) \\
&=   \operatorname{E}[X] - \beta \left(  \operatorname{E}[X^2]  -  \operatorname{E}[X]^2 \right) \\
& = \sum_{j\in\{0\}\cup\mathcal{X}} \Bigl( f_j t_j \prod_{\substack{i \in \mathcal{ ̄X}: \\ i > j}} (1 - f_{i}) \Bigr)
 - \beta \sum_{j\in\{0\}\cup\mathcal{X}} \Bigl( f_j t_j^2 \prod_{\substack{i \in \mathcal{X}: \\ i > j}} (1 - f_{i}) \Bigr)
  + \beta v(\mathcal{X})^2 \\
  &= v(\mathcal{X}; \tau) +  \beta v(\mathcal{X}; t)^2
\end{split}
\end{align}
\ifen
where $\tau_j = t_j - \beta t_j^2$. Since the first term is itself a portfolio valuation function, and the second is a monotonic transformation of one, we speculate that one of the algorithms given above could be used as a subroutine to trace out the efficient frontier by maximizing $v(\mathcal{X}; t)$ subject to the budget constraint and $v(\mathcal{X}; \tau) \geq \alpha$ for various values of $\alpha \geq 0$. 
\else
단, $\tau_j = t_j - \beta t_j^2$. 이의 첫 번째 항 그 자체가 포트폴리오 가치 함수이며, 두 번째 항은 포트폴리오 가치 함수의 단조 변환이다. 따라서 위에서 제시한 알고리즘을 서브루틴으로 사용하면 효율적 투자선(efficient frontier)을 탐색할 수 있어 보인다. 이때 다양한 $\alpha \geq 0$에 대해 예산 제약식과 $v(\mathcal{X}; \tau) \geq \alpha$ 아래서 $v(\mathcal{X}; t)$을 최대화하는 것이 알고리즘의 요점이다.
\fi

\ifen
A parametric treatment of risk aversion may align our model more closely to real-world applicant behavior. Conventional wisdom asserts that the best college application strategy combines reach, target, and safety schools in roughly equal proportion (Jeon 2015). When the application budget is small relative to the size of the admissions market, our algorithms recommend a similar approach (see the example of Subsection \ref{planetsexamplesection}). However, inspecting equation \eqref{howtotransformtj}, which discounts school utility parameters to reflect their marginal value relative to the schools already in the portfolio, reveals that low-utility schools are penalized more harshly than high utility schools in the marginal analysis. Consequentially, as the application budget grows, the optimal portfolio in our model tends to favor reach schools over safety schools. (See Figure \ref{samplemarket} for a clear illustration of this phenomenon.)
 
A potential application of our model is an econometric study that estimates students’ perceptions of college quality ($t_j$) from their observed application behavior ($\mathcal{X}$) under the assumption of utility maximization. Empirical research indicates that heterogeneous risk aversion is a significant source of variation in human decision-making, especially in the domains of educational choice and employment search (Kahneman 2011; Hartlaub and Schneider 2012; van Huizen and Alessie 2019). Therefore, an endogenous risk-aversion parameter would greatly enhance our model’s explanatory power in the econometric setting.
\else
위험 회피를 모수화한 모형은 실제 지원 행동을 더 정확하게 반영할 수 있다. ``상향·소신·안정'' 지원 학교를 대략 균일하게 분산하여 지원하는 것이 최적이란 통념이다 (전민희 2015). 입학 시장의 크기에 비해 지원 예산이 작은 경우, 본 연구에서 제시하는 알고리즘은 비슷한 전략을 추천한다 (항 \ref{planetsexamplesection}의 예제 참고). 그러나 포트폴리오에 이미 삽입한 학교를 반영한 한계 가치를 나타내도록 효용 모수를 할인하는 수식 \eqref{howtotransformtj}을 살펴보면, 효용이 높은 학교보다 효용이 낮은 학교에 부과되는 페널티가 큰 것을 알 수 있다. 그 결과는 지원 예산이 커질수록 본 모형의 최적 포트폴리오가 안정 지원 학교보다 상향 지원 학교를 편드는 것이다. (그림 \ref{samplemarket}에서 뚜렷이 보이는 현상이다.)

학생의 지원 행동($\mathcal{X}$)을 관찰하고 학생이 인식하는 대학교의 질($t_j$)을 추정하는 경제 측정학 연구에서 본 논문의 모형을 그대로 응용할 수 있다. 실무 연구 결과에 따르면 위험 회피의 다양성은 인간 의사 결정 다양성의 주된 원인이 되며, 교육 선택과 취업 활동은 특히 그렇다 (Kahneman 2011; Hartlaub와 Schneider 2012; van Huizen과 Alessie 2019). 따라서 경제 측정학 맥락에서, 위험 회피 모수를 내생 변수로 도입하면 원래 모형의 묘사력을 많이 개선할 수 있다.
\fi

\ifen \subsection{Signaling strategies and matroid constraints}\else \subsection{시그널 전략과 matroid 제약 조건}\fi
\ifen
Another direction of further research with immediate application in college admissions is to incorporate additional signaling strategies into the problem. In the Korean admissions process, the online application form contains three multiple-choice fields, labeled \emph{ga, na,} and \emph{da} for the first three letters of the Korean alphabet, which students use to indicate the colleges to which they wish to apply. Most schools appear only in one or two of the three fields. Therefore, students are restricted not only in the number of applications they can submit, but in the combinations of schools that may coincide in a feasible application portfolio. From a portfolio optimization perspective, this is a \emph{diversification constraint,} because its primary purpose is to prevent students from applying only to top-tier schools. However, in addition to overall selectivity, the three fields also indicate different evaluation schemes: The \emph{ga} and \emph{na} customarily indicate the applicant's first and second choice, and applications filed under these fields are evaluated with greater emphasis on standardized-test scores than those filed under the \emph{da} field. Some colleges appear in multiple fields and run two parallel admissions processes, each with different evaluation criteria. Therefore, if a student recognizes that she has a comparative advantage in (for example) her interview skills, she may increase her chances of admission to a competitive school by applying under the \emph{da} field. The optimal application strategy in this context can be quite subtle, and it is a subject of perennial debate on Korean social networks.

An analogous feature of the American college admissions process is known as early decision, in which at the moment of application, a student commits to enrolling in a college if admitted. In principle, students who apply with early decision may obtain a better chance of admission by signaling their eagerness to attend, but doing so weakens the college's incentive to entice the student with discretionary financial aid, altering the utility calculus.
\else
향후 연구에 두 번째 방향은 모형에 시그널 전략을 도입하는 것이다. 한국 정시 입학 과정의 온라인 지원 양식에는 가군, 나군, 다군이란 3가지 선다형 칸을 채워서 지원하는 학교를 선택한다. 대부분의 학교는 모든 3개의 칸에 등장하지 않고 그중 1~2개만 가능하다. 따라서 학생이 지원할 수 있는 학교의 `수'일 뿐만 아니라 가능한 지원 포트폴리오에 같이 들어갈 수 있는 학교의 `조합'도 제한된다. 이는 학생이 상위권 대학만 지원하지 않도록 만들어지기 때문에 다각화(diversification) 제약식으로 볼 수 있다. 그러나 대학이 어떤 군에 속하는 것은 대학 순위일뿐더러 입학 전형 유형도 의미한다. 가군과 나군은 관례적으로 학생이 선호하는 대학을 선택하는 것이며 다군보다 수능 점수의 반영 비율이 높은 편이다. 또한 군 2개에 포함되는 학교도 있으며 입학 전형이 다르면서 평행한 모집 과정 2개를 운영한다. 가령, 어떤 학생이 면접 능력에 비교 우위가 있으면 경쟁적인 학교에 다군으로 지원하면 합격 확률을 높일 수도 있다. 이런 맥락에서 최적의 지원 전략은 미묘한 문제이며 한국 SNS에서 자주 발생하는 화젯거리다.

미국 입학 과정의 유사한 특징은 조기 전형(early decision)이라고 불린다. 이는 학생이 합격하면 진학한다고 약속하고 지원하는 것을 의미한다. 조기 전형으로 지원하면 입학할 관심을 시그널할 수 있으므로 합격 확률을 높일 수 있지만, 학교가 학생을 모집하려고 장학금을 수여할 동기가 약화할 수 있으므로 학생의 효용 추정이 달라진다.
\fi


\ifen
In the integer formulation of the college application problem (Problem \ref{integernlp}), a natural way to model these signaling strategies without introducing too much complexity is to split each $c_j$ into $c_{j+}$ and $c_{j-}$. The binary decision variables are $x_{j+} = 1$ if the student applies to $c_j$ with high priority (that is, in  the \emph{ga} or \emph{na} field or with early decision) and $x_{j-} = 1$ if applying with low priority (that is, in the \emph{da} field or without early decision). Now, assuming that the differential admission probabilities $f_{j+}$ and $f_{j-}$ and utilities $t_{j+}$ and $t_{j-}$ are known, adding the logical constraints 
\begin{align}
 x_{j+} + x_{j-} \leq 1, ~~ j = 1\dots m\qquad\text{and}\qquad\sum_{j=1}^m x_{j+} \leq 1
\end{align}
completes the formulation. These are knapsack constraints; therefore, the submodular maximization algorithm of Kulik et al. (2013) can be used to obtain a $(1 - 1/e - \varepsilon)$-approximate solution to this problem. When the budget constraint is a cardinality constraint (as in Alma's problem), we note that imposing the signaling constraints yields a matroidal solution set, and Calinescu et al. (2011)'s algorithm yields an alternative solution, which has the same approximation coefficient of $1 - 1/e -\varepsilon$. %These constraints are conformable to those of the multiple knapsack problem (Kellerer et al. 2004, chap. 10).
\else
대학 지원 문제의 정수 모형(문제 \ref{integernlp})에서 위의 시그널 전략들을 어렵지 않게 도입하는 방식은 각 $c_j$를 $c_{j+}$와 $c_{j-}$로 나누는 것이다. 그다음에 학생이 $c_j$에 높은 관심을 시그널하면 (즉, 가군으로 선택하거나 조기 전형으로 지원하면) $x_{j+} = 1$이며 관심을 시그널 하지 않고 지원하면 (즉 가군 혹은 나군으로 지원하거나 조기 전형 없이 지원하면) $x_{j-} = 1$이 되는 이진 결정 변수를 정의한다. 해당 합격 확률 $f_{j+}$와 $f_{j-}$ 그리고 효용  $t_{j+}$와 $t_{j-}$가 알려져 있으면, 다음 같은 논리 제약식을 더하면 모형을 완성할 수 있다. 
\begin{align}
 x_{j+} + x_{j-} \leq 1, ~~ j = 1\dots m\qquad\text{그리고}\qquad\sum_{j=1}^m x_{j+} \leq 1
\end{align}
모두 배낭 제약식이므로 Kulik 외 (2013)의 submodular 최대화 알고리즘을 통해 이 문제의 $(1 - 1/e - \varepsilon)$-근사해를 구할 수 있다. 또한 (알마의 문제처럼) 예산 제약식이 집합 크기 제약식이 되면, 논리 제약식을 더한 모형의 해집합이 matroid가 되며, Calinescu 외 (2011)의 알고리즘이 대안 해법이 된다. 단, 그의 근사 계수는 여전히 $1 - 1/e - \varepsilon$이다. 
\fi

\ifen 
Our main results indicate that the objective function of the college application problem, though nonlinear, has similar regularity properties to a linear function: The greedy algorithm finds the maximum over a cardinality constraint, and an FPTAS exists for maximization over a knapsack constraint. For linear functions, however, the greedy algorithm is optimal over matroid constraints as well (Edmonds 1971). Thus, if the pseudolinearity of our objective function is sufficiently robust, the greedy algorithm will be an exact solution for the college application with matroid constraints, including the signaling constraints discussed above. Preliminary numerical experiments suggest that this is the case, but the class of matroids is diverse and difficult to test exhaustively. (The nestedness property does not apply automatically to matroids because in the proof of the elimination technique given in \S\,\ref{eliminationtechniquesection}, we assumed that $\mathcal{X} \cup \{k\}$ implies the feasibility of $\mathcal{Y} \cup \{k\}$ when $|\mathcal{X}| = |\mathcal{Y}|$.)
\else
본 연구의 주된 결과는 대학 지원 문제의 목적함수가 선형 함수가 아니지만, 최적화 관점에서 선형 함수와 비슷한 성질을 가진다는 것으로 해석할 수 있다. 집합 크기 제약 아래서 탐욕 해법이 정확하고, 배낭 제약 아래서 FPTAS가 존재하기 때문이다. 그런데 선형 함수는 matroid 제약 아래서 최적화하는 문제도 탐욕 해법으로 풀 수 있다 (Edmonds 1971). 따라서 대학 지원 목적함수와 선형 함수의 유사성이 적당히 깊으면 위에서 논한 matroid 제약 아래서 이를 최대화하는 문제를 탐욕 해법으로 풀 수 있을 것이다. 예비 수리 실험을 진행한 결과로 이 성질이 사실처럼 보이지만, matroid가 아주 광범위한 클래스이므로 철저한 실험이 어렵다. (\S\,\ref{eliminationtechniquesection}에서 제시된 변수 기법의 증명 과정에서 $|\mathcal{X}| = |\mathcal{Y}|$일 때 $\mathcal{X} \cup \{k\}$가 가능해면 $\mathcal{Y} \cup \{k\}$도 가능해라고 가정하기 때문에 포함 사슬 관계 성질을 matroid에 자동으로 확장할 수 있는 것이 아니다.)
\fi



\ifen \subsection{Memory-efficient dynamic programs}\else \subsection{동적 계획의 메모리 소요 절감}\fi
\ifen
Our numerical experiments suggest that the performance of the dynamic programming algorithms is bottlenecked by not computation time, but memory usage. Reducing these algorithms' storage requirements would enable us to solve considerably larger problems.

Abstractly speaking, Algorithms \ref{ellisDP1} and \ref{ellisDP3} are two-dimensional dynamic programs that represent the optimal solution by $Z[N, C]$. Here $N$ is the number of decision variables, $C$ is a constraint parameter, and $Z[n, c]$ is the optimal objective value achievable using the first $n\leq N$ decision variables when the constraint parameter is $c \leq C$. The algorithm iterates on a recursion relation that expresses $Z[n, c]$ as a function of $Z[n -1, c]$ and $Z[n -1, c']$ for some $c' \leq c$. (In Algorithm \ref{ellisDP1}, $Z$ is the maximal portfolio valuation, $N = m$, and $C =H$; in Algorithm \ref{ellisDP3},  $Z$ is the minimal application expenditures, $N = m$, and $C = |\mathcal{V}| \propto m^2 / \varepsilon$.)

When such a dynamic program is implemented using a lookup table or dictionary memoization, producing the optimal solution requires $O(NC)$ time and space. Kellerer et al. (2004, \S\,3.3) provide a technique for transforming the dynamic program $Z$ into a divide-and-conquer algorithm that produces the optimal solution in $O(N C)$ time and $O(N + C)$ space, a significant improvement. However, their technique requires the objective function to be additively separable in a certain sense that appears difficult to conform to the college application problem. 
\else
수리 실험 결과에 따르면 동적 계획 해법 성능의 병목 요소는 계산 시간이 아니라 메모리 소모량이다. 이 알고리즘의 메모리 소요를 절감하면 상당히 큰 문제를 풀 수 있다.

추상적인 관점에서, 알고리즘 \ref{ellisDP1}과(와) 알고리즘 \ref{ellisDP3}은(는) 최적해를 $Z[N, C]$로 표현하는 2차원 동적 계획이다. 이때 $N$은 결정 변수의 수, $C$는 제약식의 모수, 그리고 $Z[n, c]$는 제약 모수가 $c \leq C$일 때 첫 $n\leq N$개의 결정 변수만 이용하는 최적 목적 함숫값이다. 알고리즘의 구조는 $Z[n, c]$를 $Z[n -1, c]$와 어떤 $c' \leq c$에 대응하는 $Z[n -1, c']$의 함수로 표현하고 그 반복 관계로 탐색하는 것이다. (알고리즘 \ref{ellisDP1}에서 $Z$는 최대 포트폴리오 가치, $N = m$, 그리고 $C =H$이며, 알고리즘 \ref{ellisDP3}에서 $Z$는 최소한 지원 지출액, $N = m$, 그리고 $C= |\mathcal{V}| \propto m^2 / \varepsilon$이다.)

이러한 형태의 동적 계획을 표 혹은 사전 기록으로 구현하면 최적해를 출력하는 것은 $O(NC)$ 시간과 $O(NC)$ 공간을 소요한다. Kellerer 외 (2004, \S\,3.3)는 동적 계획 $Z$를 $O(N C)$ 시간과 $O(N + C)$ 공간 안에 최적해를 구하는 분할 정복(divide and conquer) 알고리즘으로 변환하는 일반적인 방법을 제시하며 매우 유익하다. 그러나 이를 이용할 수 있는 조건 중, 목적함수는 어떤 기술적인 의미에서 가산적으로 분할되어야 하며 이를 대학 지원 문제에 적용하기 어렵다.
\fi

\appendix
\ifen \section{Appendix} \else \section{부록} \fi

\ifen \subsection{Elementary proof of Theorem \ref{concavityinh}} \else \subsection{정리 \ref{concavityinh}의 기초적 증명} \fi \label{elementaryconcavityproof}
\begin{proof}
\ifen We will prove the equivalent expression $2 v(\mathcal{X}_h) \geq v(\mathcal{X}_{h+1}) + v(\mathcal{X}_{h-1})$. Applying Theorem \ref{nestedapplication}, we write $\mathcal{X}_h = \mathcal{X}_{h-1} \cup\{j\}$ and $\mathcal{X}_{h+1} = \mathcal{X}_{h-1} \cup\{j, k\}$. If $t_k \leq t_j$, then 
\else $2 v(\mathcal{X}_h) \geq v(\mathcal{X}_{h+1}) + v(\mathcal{X}_{h-1})$ 같은 동등한 부등식을 증명하자. 정리 \ref{nestedapplication}을(를) 적용하면 $ \mathcal{X}_h = \mathcal{X}_{h-1} \cup\{j\}$ 그리고 $\mathcal{X}_{h+1} = \mathcal{X}_{h-1} \cup\{j, k\}$으로 표현할 수 있다. $t_k \leq t_j$인 경우, \fi
\begin{align}
\begin{split}
2 v(\mathcal{X}_h) &= v(\mathcal{X}_{h-1} \cup\{j\}) + v(\mathcal{X}_{h-1} \cup\{j\}) \\
&\geq v(\mathcal{X}_{h-1} \cup\{k\}) + v(\mathcal{X}_{h-1} \cup\{j\}) \\
&= v(\mathcal{X}_{h-1} \cup\{k\}) + (1 - f_j) v(\mathcal{X}_{h-1}) + f_j \operatorname{E}[\max\{t_j, X_{h-1}\}] \\
&= v(\mathcal{X}_{h-1} \cup\{k\}) - f_j v(\mathcal{X}_{h-1}) + f_j \operatorname{E}[\max\{t_j, X_{h-1}\}] + v(\mathcal{X}_{h-1})  \\
&\geq v(\mathcal{X}_{h-1} \cup\{k\})  - f_j v(\mathcal{X}_{h-1}\cup\{k\}) + f_j \operatorname{E}[\max\{t_j, X_{h-1}\}]+ v(\mathcal{X}_{h-1})\\
&= (1 - f_j) v(\mathcal{X}_{h-1} \cup\{k\})  + f_j \operatorname{E}[\max\{t_j, X_{h-1}\}]+ v(\mathcal{X}_{h-1})\\
&=  v(\mathcal{X}_{h-1} \cup\{j, k\}) + v(\mathcal{X}_{h-1})\\
&=  v(\mathcal{X}_{h+1}) + v(\mathcal{X}_{h-1}).
\end{split} 
\end{align}
\ifen The first inequality follows from the optimality of $\mathcal{X}_h$, while the second follows from the fact that adding $k$ to $\mathcal{X}_{h-1}$ can only increase its valuation.
\else 첫 번째 부등식은 $\mathcal{X}_h$의 최적성에 따르며, 두 번째 부등식은 $\mathcal{X}_{h-1}$에 $j$를 더하면 가치가 증가할 수밖에 없기 때문이다.\fi

\ifen If $t_k \geq t_j$, then the steps are analogous:
\else  $t_k \geq t_j$인 경우는 유사하다:\fi
\begin{align}
\begin{split}
2 v(\mathcal{X}_h) &= v(\mathcal{X}_{h-1} \cup\{j\}) + v(\mathcal{X}_{h-1} \cup\{j\}) \\
&\geq v(\mathcal{X}_{h-1} \cup\{k\}) + v(\mathcal{X}_{h-1} \cup\{j\}) \\
&= (1 - f_k) v(\mathcal{X}_{h-1}) + f_k \operatorname{E}[\max\{t_k, X_{h-1}\}] +  v(\mathcal{X}_{h-1} \cup\{j\})  \\
&= v(\mathcal{X}_{h-1}) - f_k  v(\mathcal{X}_{h-1}) + f_k \operatorname{E}[\max\{t_k, X_{h-1}\}] +  v(\mathcal{X}_{h-1} \cup\{j\})  \\
&\geq v(\mathcal{X}_{h-1}) - f_k  v(\mathcal{X}_{h-1}\cup\{j\}) + f_k \operatorname{E}[\max\{t_k, X_{h-1}\}] +  v(\mathcal{X}_{h-1} \cup\{j\})  \\
&= v(\mathcal{X}_{h-1}) + (1 - f_k) v(\mathcal{X}_{h-1}\cup\{j\}) + f_k \operatorname{E}[\max\{t_k, X_{h-1}\}]  \\
&= v(\mathcal{X}_{h-1}) + v(\mathcal{X}_{h-1} \cup\{j, k\})
\end{split}\\
&= v(\mathcal{X}_{h-1})  + v(\mathcal{X}_{h+1}) \qedhere
\end{align}
\end{proof}



\iffalse

\subsection{Generalization of Theorem \ref{nestedapplication} for matroid constraints} \label{matroidnestedapplicationheader}

\begin{definition}[Matroid]
Consider a ground set $\mathcal{C}$ and a collection $\mathfrak{I}$ of subsets of $\mathcal{C}$, the system $(\mathcal{C}, \mathfrak{I})$ is called a \emph{matroid} if and only if the following three condtions hold:
\begin{enumerate}
\item $\varnothing \in \mathfrak{I}$.
\item $\mathfrak{I}$ is downward-closed: $\mathcal{X} \in \mathfrak{I}$ and $ \mathcal{Y} \subseteq \mathcal{X}$ implies $\mathcal{Y} \in \mathfrak{I}$.
\item If $\mathcal{X} \in \mathfrak{I}$ and $|\mathcal{Y}| < |\mathcal{X}|$, then there is an element $k \in \mathcal{X}$ such that $\mathcal{Y} \cup \{k\} \in \mathfrak{I}$. 
\end{enumerate}
The elements of $\mathfrak{I}$ are called the \emph{independent sets} of the matroid; subsets of $\mathcal{C}$ that are not independent are called \emph{dependent}. If $\mathcal{X}$ is an independent set such that $\mathcal{X} \cup \{j\} \notin \mathfrak{I}$ for all $j \in\mathcal{C} \setminus \mathcal{X}$, then $\mathcal{X}$ is called \emph{base} of $(\mathcal{C}, \mathfrak{I})$. %The third property implies that all bases have the same cardinality, called the matroid's \emph{rank.}
\end{definition}


\begin{problem}[\ifen Alma’s problem with matroid constraints\else 매트로이드 제약이 있는 알마의 문제\fi]
%\ifen Alma's optimal college application portfolio is given by the solution to the following combinatorial optimization problem:
%\else
%알마의 최적 대학 지원 포트폴리오는 다음 조합 최적화 문제의 최적해다.
%\fi
\begin{align}
\begin{split}
\text{maximize}\quad &  v(\mathcal{X}) = \sum_{j\in
%\{0\}\cup
\mathcal{X}} \Bigl( f_j t_j \prod_{\substack{i \in \mathcal{X}: \\ i > j}} (1 - f_{i}) \Bigr)\\
\text{subject to}\quad & \mathcal{X} \in \mathfrak{I}
\end{split}
\end{align}
where $\mathcal{C}, \mathfrak{I})$ is a matroid.
\end{problem}



\begin{definition}[Greedy algorithm for Alma's problem with matroid constraints]
At each iteration, add to $\mathcal{X}$ the school $k$ for which $v(\mathcal{X} \cup \{k\})$ is greatest and $\mathcal{X} \in $
\end{definition}

%\begin{theorem}
%
%\end{theorem}
\else



















\pagebreak
\ifen
\section*{References}
\addcontentsline{toc}{section}{References}
\else
\section*{참고문헌}
\addcontentsline{toc}{section}{참고문헌}
\fi
\noindent


% Add to bib: Gale and Shapley 1962 Roth 1982, , Azevedo and Leshno 2016 Bodoh-Creed 2020; Ashlagi and Nikzad 2020, Kapur 2021




\end{document}  

