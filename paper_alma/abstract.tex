This paper considers a novel portfolio optimization problem called the college application
problem. We show that the objective function is a nondecreasing submodular set function
and provide a quadratic-time algorithm for its maximization over a cardinality
constraint. The algorithm's validity rests on the nested structure of the optimal solution,
a property that is of independent interest. Further analysis of the optimal application strategy
helps explain perplexing findings from regression studies of college applicant behavior:
In our model, systematic differences in the risk allocations of low- and high-income students
emerge spontaneously from differences in applicants' budgets, even in the absence of risk aversion.
Also, our analysis suggests that the underwhelming effectiveness of behavioral nudges
(such as application fee waivers) designed to encourage talented low-income students to apply
to more colleges can be attributed to the fact that these interventions have typically targeted
students with small shadow costs.
